% reversoes e transposicoes
Na natureza um genoma não sofre apenas eventos de reversão ou de
transposição, ele está exposto a diversos eventos diferentes. Para
esta situação, iremos estudar o caso onde os eventos de reversão e
transposição ocorrem simultaneamente sobre um genoma.

A distância de reversão e transposição $d_{rt}(\pi, \sigma)$ entre
duas permutações $\pi$ e $\sigma$ é o número mínimo $rt$ de reversões
e transposições $\rho_{1}, \rho_{2}, \ldots, \rho_{rt}$ tal que
$\pi \rho_{1} \rho_{2} \ldots \rho_{rt} = \sigma$. Como no caso em que
os eventos ocorrem individualmente, podemos dizer, sem perda de
generalidade, que o problema da distância de reversão e transposição é
equivalente ao problema de ordenação por reversões e transposições,
que é a distância de reversão e transposição entre a permutação $\pi$ e a
permutação identidade $\iota$, denotado por $d_{rt}(\pi)$.
