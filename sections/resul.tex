% resultados
Neste capítulo apresentaremos os resultados obtidos pelos modelos
descritos no capítulo~\ref{cap:model}. A seção \ref{sec:tspec} mostra
as características do computador utilizado para executar os testes. A
seção \ref{sec:testes} descreve como os testes foram executados. A
seção \ref{sec:analise} apresenta a análise dos resultados obtidos
durante este trabalho.

\section{Especificações Técnicas}
\label{sec:tspec}
O computador utilizado para executar os testes possui as seguintes
características:

\begin{itemize}
  \item{Processador: Intel\textregistered{}~Core\texttrademark~2 Duo
  2.33GHz.}

  \item{Memória RAM: 3 GB.}
  
  \item{Sistema Operacional: Ubuntu Linux com kernel 2.6.31.}
\end{itemize}

Todos modelos de \pr{} foram implementados usando as seguintes
ferramentas:

\begin{itemize}
  \item{Sistema de programação de código
  aberto \textit{ECLiPSe-6.0}~\cite{eclipse*2009}. } 
  
 \item{Pacote proprietário para a linguagem de programação
  C++ \textit{IBM\textregistered{} ILOG\textregistered{}
  CPLEX\textregistered{} CP Optimizer v 2.3}~\cite{ilogcp*2011}.}
\end{itemize}

Todas formulações de programação linear inteira foram implementadas
usando as seguintes ferramentas:

\begin{itemize}
  \item{Sistema de programação de código
  aberto \textit{GLPK-4.35}~\cite{glpk*2010}.}
  
  \item{Pacote proprietário para a linguagem de programação
  C++ \textit{IBM\textregistered{} ILOG\textregistered{}
  CPLEX\textregistered{} Optimizer v 12.1}~\cite{ilogcplex*2011}..}
\end{itemize}

\section{Descrição dos Testes}
\label{sec:testes}
Os testes foram separados de acordo com o tamanho das
permutações. Uma instância contém o conjunto de permutações com
tamanho $n$, onde $n > 2$ devido ao fato de ser trivial fazer uma das
operações de ordenação em uma permutação com tamanho $2$. Para cada
instância, geramos $50$ permutações com tamanho $n$.

Todas instâncias foram executadas nos softwares indicados na
seção \ref{sec:tspec}. Para cada instância foi dado o tempo máximo de
$25$ horas. Fazemos a comparação dos modelos se baseando nos tempos
médios usados para resolver cada instância. Como referência usamos os
modelos de \pli{} descritos na seção \ref{sec:pli}.

\section{Análise dos Resultados}
\label{sec:analise}
As tabelas \ref{table:rev}, \ref{table:trans}, \ref{table:r_t}
apresentam os tempos médios usados para resolver cada instância dos
testes. O caractere ``-'' significa que o modelo não conseguiu
solucionar todas as permutações da instância dentro do limite de 25
horas. O caractere ``*'' significa que o modelo não conseguiu terminar
devido ao limite de memória do sistema.

Podemos observar que nos três casos que os modelos de programação por
restrições baseados na teoria COP possuem os piores tempos de execução
e os modelos baseados na teoria CSP possuem os melhores
resultados. Outro fato que podemos notar é que quanto melhor os
limitantes menor é o tempo necessário para solucionar as instâncias.

% DO NOT FORGET THE PACKAGE 'rotating'!!!!
% DO NOT FORGET THE PACKAGE 'array'!!!!
% DO NOT FORGET THE PACKAGE 'multirow'!!!!
\begin{sidewaystable}[!ht]
  \caption[Modelos de Ordenação por Reversões.]{Tempo médio (em segundos) para o modelo de ordernação por reversões. O caractere ``-'' significa que o modelo não conseguiu terminar o conjunto de testes dentro do limite de 25 horas. O caractere ``*'' significa que o modelo não conseguir terminar devido ao limite de memória do sistema.}
  \label{table:rev}
  \begin{center}
  \scalebox{0.670000}{
    \begin{tabular}{| r | >{\raggedleft\arraybackslash}p{1.8cm} | >{\raggedleft\arraybackslash}p{1.8cm} | >{\raggedleft\arraybackslash}p{1.8cm} | >{\raggedleft\arraybackslash}p{1.8cm} | >{\raggedleft\arraybackslash}p{1.8cm} | >{\raggedleft\arraybackslash}p{1.8cm} | >{\raggedleft\arraybackslash}p{1.8cm} | >{\raggedleft\arraybackslash}p{1.8cm} | >{\raggedleft\arraybackslash}p{1.8cm} | >{\raggedleft\arraybackslash}p{1.8cm} | >{\raggedleft\arraybackslash}p{1.8cm} | >{\raggedleft\arraybackslash}p{1.8cm} | >{\raggedleft\arraybackslash}p{1.8cm} | >{\raggedleft\arraybackslash}p{1.8cm} |}
      \hline
      \multicolumn{15}{|c|}{\textbf{Modelos de Ordenação por Reversões}} \\
      \hline
      \multirow{4}{*}{\textbf{size}} & \multicolumn{12}{c|}{\textbf{CP}} & \multicolumn{2}{c|}{\textbf{ILP}} \\
      \cline{02-15}
        & \multicolumn{6}{c|}{\textbf{ECLiPSe}} & \multicolumn{6}{c|}{\textbf{ILOG CP}} & \centering\multirow{3}{*}{\textbf{GLPK}} & \multirow{3}{1.8cm}{\centering\textbf{ILOG CPLEX}} \\
      \cline{02-13}
        & \multicolumn{3}{c|}{\textbf{CSP}} & \multicolumn{3}{c|}{\textbf{COP}} & \multicolumn{3}{c|}{\textbf{CSP}} & \multicolumn{3}{c|}{\textbf{COP}} &  & \\
      \cline{02-13}
        & \centering\textbf{~def~} & \centering\textbf{~rev\_br~} & \centering\textbf{~rev\_cg~} & \centering\textbf{~def~} & \centering\textbf{~rev\_br~} & \centering\textbf{~rev\_cg~} & \centering\textbf{~def~} & \centering\textbf{~rev\_br~} & \centering\textbf{~rev\_cg~} & \centering\textbf{~def~} & \centering\textbf{~rev\_br~} & \centering\textbf{~rev\_cg~} &  & \\
      \hline
      ~3~ & ~0.008~ & ~0.003~ & ~0.013~ & ~0.004~ & ~0.004~ & ~0.003~ & ~0.001~ & ~0.001~ & ~0.001~ & ~0.001~ & ~0.001~ & ~0.001~ & ~0.001~ & ~0.001~ \\
      ~4~ & ~0.368~ & ~0.339~ & ~1.490~ & ~0.025~ & ~0.004~ & ~0.003~ & ~0.001~ & ~0.001~ & ~0.001~ & ~0.001~ & ~0.001~ & ~0.001~ & ~0.001~ & ~0.002~ \\
      ~5~ & ~31.676~ & ~39.424~ & ~93.984~ & ~0.531~ & ~0.014~ & ~0.008~ & ~0.019~ & ~0.018~ & ~0.019~ & ~0.001~ & ~0.006~ & ~0.004~ & ~0.176~ & ~0.039~ \\
      ~6~ & ~-~ & ~-~ & ~*~ & ~9.001~ & ~0.032~ & ~0.014~ & ~0.144~ & ~0.148~ & ~0.126~ & ~0.001~ & ~0.023~ & ~0.020~ & ~2.118~ & ~0.653~ \\
      ~7~ & ~-~ & ~-~ & ~-~ & ~442.825~ & ~0.179~ & ~0.072~ & ~1.962~ & ~1.807~ & ~1.331~ & ~0.007~ & ~0.156~ & ~0.225~ & ~3.896~ & ~21.911~ \\
      ~8~ & ~-~ & ~-~ & ~-~ & ~-~ & ~1.075~ & ~0.449~ & ~25.831~ & ~26.275~ & ~30.244~ & ~0.001~ & ~0.642~ & ~2.757~ & ~3.006~ & ~280.528~ \\
      ~9~ & ~-~ & ~-~ & ~-~ & ~-~ & ~5.015~ & ~1.401~ & ~503.421~ & ~473.005~ & ~405.242~ & ~0.003~ & ~11.416~ & ~54.560~ & ~-~ & ~5.161~ \\
      ~10~ & ~-~ & ~-~ & ~-~ & ~-~ & ~51.800~ & ~8.639~ & ~-~ & ~-~ & ~-~ & ~*~ & ~*~ & ~*~ & ~-~ & ~-~ \\
      ~11~ & ~-~ & ~-~ & ~-~ & ~-~ & ~1664.691~ & ~194.266~ & ~-~ & ~-~ & ~-~ & ~-~ & ~-~ & ~-~ & ~-~ & ~-~ \\
      ~12~ & ~-~ & ~-~ & ~-~ & ~-~ & ~-~ & ~671.979~ & ~-~ & ~-~ & ~-~ & ~-~ & ~-~ & ~-~ & ~-~ & ~-~ \\
      ~13~ & ~-~ & ~-~ & ~-~ & ~-~ & ~-~ & ~*~ & ~-~ & ~-~ & ~-~ & ~-~ & ~-~ & ~-~ & ~-~ & ~-~ \\
      \hline
    \end{tabular}
  }
  \end{center}
\end{sidewaystable}

% DO NOT FORGET THE PACKAGE 'rotating'!!!!
% DO NOT FORGET THE PACKAGE 'array'!!!!
% DO NOT FORGET THE PACKAGE 'multirow'!!!!
\begin{sidewaystable}[!ht]
  \caption[Modelos de Ordenação por Transposições.]{Tempo médio (em segundos) para o modelo de ordernação por transposições. O caractere ``-'' significa que o modelo não conseguiu terminar o conjunto de testes dentro do limite de 25 horas.}
  \label{table:trans}
  \begin{center}
  \scalebox{0.670000}{
    \begin{tabular}{| r | >{\raggedleft\arraybackslash}p{1.8cm} | >{\raggedleft\arraybackslash}p{1.8cm} | >{\raggedleft\arraybackslash}p{1.8cm} | >{\raggedleft\arraybackslash}p{1.8cm} | >{\raggedleft\arraybackslash}p{1.8cm} | >{\raggedleft\arraybackslash}p{1.8cm} | >{\raggedleft\arraybackslash}p{1.8cm} | >{\raggedleft\arraybackslash}p{1.8cm} | >{\raggedleft\arraybackslash}p{1.8cm} | >{\raggedleft\arraybackslash}p{1.8cm} | >{\raggedleft\arraybackslash}p{1.8cm} | >{\raggedleft\arraybackslash}p{1.8cm} | >{\raggedleft\arraybackslash}p{1.8cm} | >{\raggedleft\arraybackslash}p{1.8cm} |}
      \hline
      \multicolumn{15}{|c|}{\textbf{Modelos de Ordenação por Transposições}} \\
      \hline
      \multirow{4}{*}{\textbf{size}} & \multicolumn{12}{c|}{\textbf{CP}} & \multicolumn{2}{c|}{\textbf{ILP}} \\
      \cline{02-15}
        & \multicolumn{6}{c|}{\textbf{ECLiPSe}} & \multicolumn{6}{c|}{\textbf{ILOG CP}} & \centering\multirow{3}{*}{\textbf{GLPK}} & \multirow{3}{1.8cm}{\centering\textbf{ILOG CPLEX}} \\
      \cline{02-13}
        & \multicolumn{3}{c|}{\textbf{COP}} & \multicolumn{3}{c|}{\textbf{CSP}} & \multicolumn{3}{c|}{\textbf{COP}} & \multicolumn{3}{c|}{\textbf{CSP}} &  & \\
      \cline{02-13}
        & \centering\textbf{~def~} & \centering\textbf{~tra\_br~} & \centering\textbf{~tra\_cg~} & \centering\textbf{~def~} & \centering\textbf{~tra\_br~} & \centering\textbf{~tra\_cg~} & \centering\textbf{~def~} & \centering\textbf{~tra\_br~} & \centering\textbf{~tra\_cg~} & \centering\textbf{~def~} & \centering\textbf{~tra\_br~} & \centering\textbf{~tra\_cg~} &  & \\
      \hline
      ~3~ & ~0.005~ & ~0.006~ & ~0.004~ & ~0.003~ & ~0.002~ & ~0.003~ & ~0.001~ & ~0.001~ & ~0.001~ & ~0.001~ & ~0.001~ & ~0.001~ & ~0.001~ & ~0.001~ \\
      ~4~ & ~0.479~ & ~1.557~ & ~0.049~ & ~0.012~ & ~0.004~ & ~0.003~ & ~0.004~ & ~0.001~ & ~0.001~ & ~0.001~ & ~0.001~ & ~0.001~ & ~0.001~ & ~0.005~ \\
      ~5~ & ~174.445~ & ~1275.804~ & ~2.922~ & ~0.318~ & ~0.020~ & ~0.005~ & ~0.023~ & ~0.009~ & ~0.004~ & ~0.001~ & ~0.008~ & ~0.008~ & ~0.376~ & ~0.054~ \\
      ~6~ & ~-~ & ~-~ & ~33.311~ & ~15.957~ & ~0.185~ & ~0.009~ & ~0.321~ & ~0.117~ & ~0.072~ & ~0.001~ & ~0.035~ & ~0.041~ & ~3.948~ & ~2.308~ \\
      ~7~ & ~-~ & ~-~ & ~-~ & ~394.574~ & ~0.519~ & ~0.015~ & ~1.747~ & ~1.313~ & ~0.321~ & ~0.005~ & ~0.112~ & ~0.173~ & ~5.558~ & ~66.497~ \\
      ~8~ & ~-~ & ~-~ & ~-~ & ~-~ & ~7.254~ & ~0.069~ & ~29.714~ & ~45.806~ & ~27.883~ & ~0.002~ & ~0.651~ & ~5.895~ & ~-~ & ~-~ \\
      ~9~ & ~-~ & ~-~ & ~-~ & ~-~ & ~99.152~ & ~0.365~ & ~641.954~ & ~838.988~ & ~320.615~ & ~0.004~ & ~2.285~ & ~53.673~ & ~-~ & ~-~ \\
      ~10~ & ~-~ & ~-~ & ~-~ & ~-~ & ~-~ & ~5.162~ & ~-~ & ~-~ & ~-~ & ~-~ & ~-~ & ~-~ & ~-~ & ~-~ \\
      ~11~ & ~-~ & ~-~ & ~-~ & ~-~ & ~-~ & ~33.122~ & ~-~ & ~-~ & ~-~ & ~-~ & ~-~ & ~-~ & ~-~ & ~-~ \\
      ~12~ & ~-~ & ~-~ & ~-~ & ~-~ & ~-~ & ~43.893~ & ~-~ & ~-~ & ~-~ & ~-~ & ~-~ & ~-~ & ~-~ & ~-~ \\
      ~13~ & ~-~ & ~-~ & ~-~ & ~-~ & ~-~ & ~521.840~ & ~-~ & ~-~ & ~-~ & ~-~ & ~-~ & ~-~ & ~-~ & ~-~ \\
      ~14~ & ~-~ & ~-~ & ~-~ & ~-~ & ~-~ & ~1223.512~ & ~-~ & ~-~ & ~-~ & ~-~ & ~-~ & ~-~ & ~-~ & ~-~ \\
      \hline
    \end{tabular}
  }
  \end{center}
\end{sidewaystable}

% DO NOT FORGET THE PACKAGE 'rotating'!!!!
% DO NOT FORGET THE PACKAGE 'array'!!!!
% DO NOT FORGET THE PACKAGE 'multirow'!!!!
\begin{sidewaystable}[!ht]
  \caption[Modelos de Ordenação por Reversões e Transposições.]{Tempo médio (em segundos) para o modelo de ordernação por reversões e transposições. O caractere ``-'' significa que o modelo não conseguiu terminar o conjunto de testes dentro do limite de 25 horas. O caractere ``*'' significa que o modelo não conseguir terminar devido ao limite de memória do sistema.}
  \label{table:r_t}
  \begin{center}
  \scalebox{0.520000}{
    \begin{tabular}{| r | >{\raggedleft\arraybackslash}p{1.8cm} | >{\raggedleft\arraybackslash}p{1.8cm} | >{\raggedleft\arraybackslash}p{1.8cm} | >{\raggedleft\arraybackslash}p{1.8cm} | >{\raggedleft\arraybackslash}p{1.8cm} | >{\raggedleft\arraybackslash}p{1.8cm} | >{\raggedleft\arraybackslash}p{1.8cm} | >{\raggedleft\arraybackslash}p{1.8cm} | >{\raggedleft\arraybackslash}p{1.8cm} | >{\raggedleft\arraybackslash}p{1.8cm} | >{\raggedleft\arraybackslash}p{1.8cm} | >{\raggedleft\arraybackslash}p{1.8cm} | >{\raggedleft\arraybackslash}p{1.8cm} | >{\raggedleft\arraybackslash}p{1.8cm} | >{\raggedleft\arraybackslash}p{1.8cm} | >{\raggedleft\arraybackslash}p{1.8cm} | >{\raggedleft\arraybackslash}p{1.8cm} | >{\raggedleft\arraybackslash}p{1.8cm} |}
      \hline
      \multicolumn{19}{|c|}{\textbf{Modelos de Ordenação por Reversões e Transposições}} \\
      \hline
      \multirow{4}{*}{\textbf{size}} & \multicolumn{16}{c|}{\textbf{CP}} & \multicolumn{2}{c|}{\textbf{ILP}} \\
      \cline{02-19}
        & \multicolumn{8}{c|}{\textbf{ECLiPSe}} & \multicolumn{8}{c|}{\textbf{ILOG CP}} & \centering\multirow{3}{*}{\textbf{GLPK}} & \multirow{3}{1.8cm}{\centering\textbf{ILOG CPLEX}} \\
      \cline{02-17}
        & \multicolumn{4}{c|}{\textbf{CSP}} & \multicolumn{4}{c|}{\textbf{COP}} & \multicolumn{4}{c|}{\textbf{CSP}} & \multicolumn{4}{c|}{\textbf{COP}} &  & \\
      \cline{02-17}
        & \centering\textbf{~def~} & \centering\textbf{~r\_t\_br~} & \centering\textbf{~r\_t\_bc~} & \centering\textbf{~r\_t\_cc~} & \centering\textbf{~def~} & \centering\textbf{~r\_t\_br~} & \centering\textbf{~r\_t\_bc~} & \centering\textbf{~r\_t\_cc~} & \centering\textbf{~def~} & \centering\textbf{~r\_t\_br~} & \centering\textbf{~r\_t\_bc~} & \centering\textbf{~r\_t\_cc~} & \centering\textbf{~def~} & \centering\textbf{~r\_t\_br~} & \centering\textbf{~r\_t\_bc~} & \centering\textbf{~r\_t\_cc~} &  & \\
      \hline
      ~3~ & ~0.035~ & ~0.010~ & ~0.004~ & ~0.006~ & ~0.004~ & ~0.005~ & ~0.004~ & ~0.008~ & ~0.001~ & ~0.001~ & ~0.001~ & ~0.001~ & ~0.001~ & ~0.001~ & ~0.001~ & ~0.001~ & ~0.001~ & ~0.001~ \\
      ~4~ & ~6.434~ & ~10.333~ & ~0.248~ & ~0.687~ & ~0.022~ & ~0.026~ & ~0.035~ & ~0.081~ & ~0.002~ & ~0.002~ & ~0.002~ & ~0.003~ & ~0.001~ & ~0.001~ & ~0.001~ & ~0.001~ & ~0.008~ & ~0.008~ \\
      ~5~ & ~-~ & ~-~ & ~30.824~ & ~64.170~ & ~0.379~ & ~0.414~ & ~0.680~ & ~2.066~ & ~0.019~ & ~0.020~ & ~0.022~ & ~0.023~ & ~0.001~ & ~0.001~ & ~0.001~ & ~0.001~ & ~0.466~ & ~0.059~ \\
      ~6~ & ~-~ & ~-~ & ~519.803~ & ~-~ & ~11.566~ & ~12.776~ & ~21.264~ & ~89.760~ & ~0.340~ & ~0.327~ & ~0.319~ & ~0.319~ & ~0.001~ & ~0.001~ & ~0.001~ & ~0.001~ & ~4.642~ & ~1.374~ \\
      ~7~ & ~-~ & ~-~ & ~-~ & ~-~ & ~400.904~ & ~441.946~ & ~792.422~ & ~*~ & ~2.052~ & ~2.062~ & ~2.066~ & ~2.088~ & ~0.004~ & ~0.004~ & ~0.002~ & ~0.004~ & ~5.094~ & ~82.241~ \\
      ~8~ & ~-~ & ~-~ & ~-~ & ~-~ & ~-~ & ~-~ & ~-~ & ~-~ & ~11.894~ & ~11.727~ & ~12.367~ & ~12.369~ & ~0.002~ & ~0.004~ & ~0.004~ & ~0.004~ & ~-~ & ~82.220~ \\
      ~9~ & ~-~ & ~-~ & ~-~ & ~-~ & ~-~ & ~-~ & ~-~ & ~-~ & ~310.012~ & ~304.331~ & ~331.275~ & ~328.956~ & ~0.006~ & ~0.006~ & ~0.004~ & ~0.005~ & ~-~ & ~-~ \\
      ~10~ & ~-~ & ~-~ & ~-~ & ~-~ & ~-~ & ~-~ & ~-~ & ~-~ & ~-~ & ~-~ & ~-~ & ~-~ & ~0.012~ & ~0.010~ & ~0.010~ & ~0.011~ & ~-~ & ~-~ \\
      \hline
    \end{tabular}
  }
  \end{center}
\end{sidewaystable}


\subsection{Ordenação por Reversões}
\label{subsec:analise_rev}
Nos modelos de ordenação por reversões, podemos notar que foi o único
modelo no qual não foi possível solucionar as permutações devido à
falta de memória.




%%%%%%%% CSP vs COP %%%%%%%%%%%
