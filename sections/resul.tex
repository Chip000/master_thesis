% resultados
Neste capítulo apresentaremos os resultados obtidos pelos modelos
descritos no capítulo~\ref{cap:model}. A seção \ref{sec:tspec} mostra
as características do computador utilizado para executar os testes. A
seção \ref{sec:testes} descreve como os testes foram executados. A
seção \ref{sec:analise} apresenta a análise dos resultados obtidos
durante este trabalho.

\section{Especificações Técnicas}
\label{sec:tspec}
O computador utilizado para executar os testes possui as seguintes
características:

\begin{itemize}
  \item{Processador: Intel\textregistered{}~Core\texttrademark~2 Duo
  2.33GHz.}

  \item{Memória RAM: 3 GB.}
  
  \item{Sistema Operacional: Ubuntu Linux com kernel 2.6.31.}
\end{itemize}

Todos modelos de \pr{} foram implementados usando as seguintes
ferramentas:

\begin{itemize}
  \item{Sistema de programação de código aberto\footnote{Em
  inglês \textit{Open
  Source}.} \textit{ECLiPSe-6.0}~\cite{eclipse*2009}. }
  
 \item{Pacote proprietário para a linguagem de programação
  C++ \textit{IBM\textregistered{} ILOG\textregistered{}
  CPLEX\textregistered{} CP Optimizer v 2.3}~\cite{ilogcp*2011}.}
\end{itemize}

Todas formulações de programação linear inteira foram implementadas
usando as seguintes ferramentas:

\begin{itemize}
  \item{Sistema de programação de código
  aberto \textit{GLPK-4.35}~\cite{glpk*2010}.}
  
  \item{Pacote proprietário para a linguagem de programação
  C++ \textit{IBM\textregistered{} ILOG\textregistered{}
  CPLEX\textregistered{} Optimizer v 12.1}~\cite{ilogcplex*2011}..}
\end{itemize}

\section{Descrição dos Testes}
\label{sec:testes}
Os testes foram separados de acordo com o tamanho das
permutações. Uma instância contém o conjunto de permutações com
tamanho $n$, onde $n > 2$ devido ao fato de ser trivial fazer uma das
operações de ordenação em uma permutação com tamanho $2$. Para cada
instância, geramos $50$ permutações com tamanho $n$.

Todas instâncias foram executadas nos softwares indicados na
seção \ref{sec:tspec}. Para cada instância foi dado o tempo máximo de
$25$ horas. Fazemos a comparação dos modelos se baseando nos tempos
médios usados para resolver cada instância. Como referência usamos os
modelos de \pli{} descritos na seção \ref{sec:pli}.

\section{Análise dos Resultados}
\label{sec:analise}
As tabelas \ref{table:rev}, \ref{table:trans}, \ref{table:r_t}
apresentam os tempos médios usados para resolver cada instância dos
testes. O caractere ``-'' significa que o modelo não conseguiu
solucionar todas as permutações da instância dentro do limite de 25
horas. O caractere ``*'' significa que o modelo não conseguiu terminar
devido ao limite de memória do sistema.

Podemos observar que nos três casos que os modelos de programação por
restrições baseados na teoria COP possuem os piores tempos de execução
e os modelos baseados na teoria CSP possuem os melhores
resultados. 

Também podemos notar que quanto melhor os limitantes menor é o tempo
necessário para solucionar as instâncias, conseguindo resolver mais
instâncias com permutações ``maiores''\footnote{Apesar que nenhum
modelo conseguiu resolver instâncias com permutações de tamanho $n >
14$, dentro do tempo limite de $25$ horas.}. Isto ocorre pela redução
do conjunto solução para o problema.

\subsection{Ordenação por Reversões}
\label{subsec:analise_rev}
Nos modelos de ordenação por reversões, tabela \ref{table:rev},
podemos notar que alguns modelos não conseguiram solucionar as
permutações devido ao limite de memória do sistema. Isto ocorreu com
as permutações de tamanho $n = 10$ usando os três limitantes nos
modelos baseados na teoria CSP desenvolvido para o \textit{ILOG CP},
com as permutações de tamanho $n = 13$ para o
limitante \textit{rev\_cg\_csp} no modelo baseado na teoria CSP e com
as permutações de tamanho $n = 6$ para o
limitante \textit{rev\_cg\_cop} no modelo baseado na teoria COP
desenvolvidos para o \textit{ECLiPSe}.

No \textit{ECLiPSe}, é possível observar que nos modelos baseados na
teoria COP, quanto melhor o limitante maior é o tempo necessário para
resolver as instâncias. Isto ocorre devido ao tamanho do espaço de
busca dos modelos COP. 

Outro fator é que os melhores limitantes usam algoritmos mais
complexos, necessitando de mais tempo para calcular os valores dos
limitantes. Isto pode ser notado nos modelos baseados na teoria CSP
desenvolvidos para o \textit{ILOG CP}.

Nenhum modelo de ordenação por reversões conseguiu solucionar as
instâncias com permutações de tamanho $n > 13$ dentro do tempo limite
de $25$ horas.

% DO NOT FORGET THE PACKAGE 'rotating'!!!!
% DO NOT FORGET THE PACKAGE 'array'!!!!
% DO NOT FORGET THE PACKAGE 'multirow'!!!!
\begin{sidewaystable}[!ht]
  \caption[Modelos de Ordenação por Reversões.]{Tempo médio (em segundos) para o modelo de ordernação por reversões. O caractere ``-'' significa que o modelo não conseguiu terminar o conjunto de testes dentro do limite de 25 horas. O caractere ``*'' significa que o modelo não conseguir terminar devido ao limite de memória do sistema.}
  \label{table:rev}
  \begin{center}
  \scalebox{0.670000}{
    \begin{tabular}{| r | >{\raggedleft\arraybackslash}p{1.8cm} | >{\raggedleft\arraybackslash}p{1.8cm} | >{\raggedleft\arraybackslash}p{1.8cm} | >{\raggedleft\arraybackslash}p{1.8cm} | >{\raggedleft\arraybackslash}p{1.8cm} | >{\raggedleft\arraybackslash}p{1.8cm} | >{\raggedleft\arraybackslash}p{1.8cm} | >{\raggedleft\arraybackslash}p{1.8cm} | >{\raggedleft\arraybackslash}p{1.8cm} | >{\raggedleft\arraybackslash}p{1.8cm} | >{\raggedleft\arraybackslash}p{1.8cm} | >{\raggedleft\arraybackslash}p{1.8cm} | >{\raggedleft\arraybackslash}p{1.8cm} | >{\raggedleft\arraybackslash}p{1.8cm} |}
      \hline
      \multicolumn{15}{|c|}{\textbf{Modelos de Ordenação por Reversões}} \\
      \hline
      \multirow{4}{*}{\textbf{size}} & \multicolumn{12}{c|}{\textbf{CP}} & \multicolumn{2}{c|}{\textbf{ILP}} \\
      \cline{02-15}
        & \multicolumn{6}{c|}{\textbf{ECLiPSe}} & \multicolumn{6}{c|}{\textbf{ILOG CP}} & \centering\multirow{3}{*}{\textbf{GLPK}} & \multirow{3}{1.8cm}{\centering\textbf{ILOG CPLEX}} \\
      \cline{02-13}
        & \multicolumn{3}{c|}{\textbf{CSP}} & \multicolumn{3}{c|}{\textbf{COP}} & \multicolumn{3}{c|}{\textbf{CSP}} & \multicolumn{3}{c|}{\textbf{COP}} &  & \\
      \cline{02-13}
        & \centering\textbf{~def~} & \centering\textbf{~rev\_br~} & \centering\textbf{~rev\_cg~} & \centering\textbf{~def~} & \centering\textbf{~rev\_br~} & \centering\textbf{~rev\_cg~} & \centering\textbf{~def~} & \centering\textbf{~rev\_br~} & \centering\textbf{~rev\_cg~} & \centering\textbf{~def~} & \centering\textbf{~rev\_br~} & \centering\textbf{~rev\_cg~} &  & \\
      \hline
      ~3~ & ~0.008~ & ~0.003~ & ~0.013~ & ~0.004~ & ~0.004~ & ~0.003~ & ~0.001~ & ~0.001~ & ~0.001~ & ~0.001~ & ~0.001~ & ~0.001~ & ~0.001~ & ~0.001~ \\
      ~4~ & ~0.368~ & ~0.339~ & ~1.490~ & ~0.025~ & ~0.004~ & ~0.003~ & ~0.001~ & ~0.001~ & ~0.001~ & ~0.001~ & ~0.001~ & ~0.001~ & ~0.001~ & ~0.002~ \\
      ~5~ & ~31.676~ & ~39.424~ & ~93.984~ & ~0.531~ & ~0.014~ & ~0.008~ & ~0.019~ & ~0.018~ & ~0.019~ & ~0.001~ & ~0.006~ & ~0.004~ & ~0.176~ & ~0.039~ \\
      ~6~ & ~-~ & ~-~ & ~*~ & ~9.001~ & ~0.032~ & ~0.014~ & ~0.144~ & ~0.148~ & ~0.126~ & ~0.001~ & ~0.023~ & ~0.020~ & ~2.118~ & ~0.653~ \\
      ~7~ & ~-~ & ~-~ & ~-~ & ~442.825~ & ~0.179~ & ~0.072~ & ~1.962~ & ~1.807~ & ~1.331~ & ~0.007~ & ~0.156~ & ~0.225~ & ~3.896~ & ~21.911~ \\
      ~8~ & ~-~ & ~-~ & ~-~ & ~-~ & ~1.075~ & ~0.449~ & ~25.831~ & ~26.275~ & ~30.244~ & ~0.001~ & ~0.642~ & ~2.757~ & ~3.006~ & ~280.528~ \\
      ~9~ & ~-~ & ~-~ & ~-~ & ~-~ & ~5.015~ & ~1.401~ & ~503.421~ & ~473.005~ & ~405.242~ & ~0.003~ & ~11.416~ & ~54.560~ & ~-~ & ~5.161~ \\
      ~10~ & ~-~ & ~-~ & ~-~ & ~-~ & ~51.800~ & ~8.639~ & ~-~ & ~-~ & ~-~ & ~*~ & ~*~ & ~*~ & ~-~ & ~-~ \\
      ~11~ & ~-~ & ~-~ & ~-~ & ~-~ & ~1664.691~ & ~194.266~ & ~-~ & ~-~ & ~-~ & ~-~ & ~-~ & ~-~ & ~-~ & ~-~ \\
      ~12~ & ~-~ & ~-~ & ~-~ & ~-~ & ~-~ & ~671.979~ & ~-~ & ~-~ & ~-~ & ~-~ & ~-~ & ~-~ & ~-~ & ~-~ \\
      ~13~ & ~-~ & ~-~ & ~-~ & ~-~ & ~-~ & ~*~ & ~-~ & ~-~ & ~-~ & ~-~ & ~-~ & ~-~ & ~-~ & ~-~ \\
      \hline
    \end{tabular}
  }
  \end{center}
\end{sidewaystable}


\subsection{Ordenação por Transposições}
\label{subsec:analise_tra}
Nos modelos de ordenação por transposições, tabela \ref{table:trans},
os modelos baseados na teoria COP que usam o
limitante \textit{tra\_br\_cop} apresentaram os piores resultados.

Similar ao modelo de ordenação por reversões, nos modelos baseados na
teoria CSP desenvolvidos para o \textit{ILOG CP}, quanto melhor o
limitante maior é o tempo necessário para resolver as instâncias.

Nenhum modelo de ordenação por transposições conseguiu solucionar as
instâncias com permutações de tamanho $n > 14$ dentro do tempo limite
de $25$ horas.

% DO NOT FORGET THE PACKAGE 'rotating'!!!!
% DO NOT FORGET THE PACKAGE 'array'!!!!
% DO NOT FORGET THE PACKAGE 'multirow'!!!!
\begin{sidewaystable}[!ht]
  \caption[Modelos de Ordenação por Transposições.]{Tempo médio (em segundos) para o modelo de ordernação por transposições. O caractere ``-'' significa que o modelo não conseguiu terminar o conjunto de testes dentro do limite de 25 horas.}
  \label{table:trans}
  \begin{center}
  \scalebox{0.670000}{
    \begin{tabular}{| r | >{\raggedleft\arraybackslash}p{1.8cm} | >{\raggedleft\arraybackslash}p{1.8cm} | >{\raggedleft\arraybackslash}p{1.8cm} | >{\raggedleft\arraybackslash}p{1.8cm} | >{\raggedleft\arraybackslash}p{1.8cm} | >{\raggedleft\arraybackslash}p{1.8cm} | >{\raggedleft\arraybackslash}p{1.8cm} | >{\raggedleft\arraybackslash}p{1.8cm} | >{\raggedleft\arraybackslash}p{1.8cm} | >{\raggedleft\arraybackslash}p{1.8cm} | >{\raggedleft\arraybackslash}p{1.8cm} | >{\raggedleft\arraybackslash}p{1.8cm} | >{\raggedleft\arraybackslash}p{1.8cm} | >{\raggedleft\arraybackslash}p{1.8cm} |}
      \hline
      \multicolumn{15}{|c|}{\textbf{Modelos de Ordenação por Transposições}} \\
      \hline
      \multirow{4}{*}{\textbf{size}} & \multicolumn{12}{c|}{\textbf{CP}} & \multicolumn{2}{c|}{\textbf{ILP}} \\
      \cline{02-15}
        & \multicolumn{6}{c|}{\textbf{ECLiPSe}} & \multicolumn{6}{c|}{\textbf{ILOG CP}} & \centering\multirow{3}{*}{\textbf{GLPK}} & \multirow{3}{1.8cm}{\centering\textbf{ILOG CPLEX}} \\
      \cline{02-13}
        & \multicolumn{3}{c|}{\textbf{COP}} & \multicolumn{3}{c|}{\textbf{CSP}} & \multicolumn{3}{c|}{\textbf{COP}} & \multicolumn{3}{c|}{\textbf{CSP}} &  & \\
      \cline{02-13}
        & \centering\textbf{~def~} & \centering\textbf{~tra\_br~} & \centering\textbf{~tra\_cg~} & \centering\textbf{~def~} & \centering\textbf{~tra\_br~} & \centering\textbf{~tra\_cg~} & \centering\textbf{~def~} & \centering\textbf{~tra\_br~} & \centering\textbf{~tra\_cg~} & \centering\textbf{~def~} & \centering\textbf{~tra\_br~} & \centering\textbf{~tra\_cg~} &  & \\
      \hline
      ~3~ & ~0.005~ & ~0.006~ & ~0.004~ & ~0.003~ & ~0.002~ & ~0.003~ & ~0.001~ & ~0.001~ & ~0.001~ & ~0.001~ & ~0.001~ & ~0.001~ & ~0.001~ & ~0.001~ \\
      ~4~ & ~0.479~ & ~1.557~ & ~0.049~ & ~0.012~ & ~0.004~ & ~0.003~ & ~0.004~ & ~0.001~ & ~0.001~ & ~0.001~ & ~0.001~ & ~0.001~ & ~0.001~ & ~0.005~ \\
      ~5~ & ~174.445~ & ~1275.804~ & ~2.922~ & ~0.318~ & ~0.020~ & ~0.005~ & ~0.023~ & ~0.009~ & ~0.004~ & ~0.001~ & ~0.008~ & ~0.008~ & ~0.376~ & ~0.054~ \\
      ~6~ & ~-~ & ~-~ & ~33.311~ & ~15.957~ & ~0.185~ & ~0.009~ & ~0.321~ & ~0.117~ & ~0.072~ & ~0.001~ & ~0.035~ & ~0.041~ & ~3.948~ & ~2.308~ \\
      ~7~ & ~-~ & ~-~ & ~-~ & ~394.574~ & ~0.519~ & ~0.015~ & ~1.747~ & ~1.313~ & ~0.321~ & ~0.005~ & ~0.112~ & ~0.173~ & ~5.558~ & ~66.497~ \\
      ~8~ & ~-~ & ~-~ & ~-~ & ~-~ & ~7.254~ & ~0.069~ & ~29.714~ & ~45.806~ & ~27.883~ & ~0.002~ & ~0.651~ & ~5.895~ & ~-~ & ~-~ \\
      ~9~ & ~-~ & ~-~ & ~-~ & ~-~ & ~99.152~ & ~0.365~ & ~641.954~ & ~838.988~ & ~320.615~ & ~0.004~ & ~2.285~ & ~53.673~ & ~-~ & ~-~ \\
      ~10~ & ~-~ & ~-~ & ~-~ & ~-~ & ~-~ & ~5.162~ & ~-~ & ~-~ & ~-~ & ~-~ & ~-~ & ~-~ & ~-~ & ~-~ \\
      ~11~ & ~-~ & ~-~ & ~-~ & ~-~ & ~-~ & ~33.122~ & ~-~ & ~-~ & ~-~ & ~-~ & ~-~ & ~-~ & ~-~ & ~-~ \\
      ~12~ & ~-~ & ~-~ & ~-~ & ~-~ & ~-~ & ~43.893~ & ~-~ & ~-~ & ~-~ & ~-~ & ~-~ & ~-~ & ~-~ & ~-~ \\
      ~13~ & ~-~ & ~-~ & ~-~ & ~-~ & ~-~ & ~521.840~ & ~-~ & ~-~ & ~-~ & ~-~ & ~-~ & ~-~ & ~-~ & ~-~ \\
      ~14~ & ~-~ & ~-~ & ~-~ & ~-~ & ~-~ & ~1223.512~ & ~-~ & ~-~ & ~-~ & ~-~ & ~-~ & ~-~ & ~-~ & ~-~ \\
      \hline
    \end{tabular}
  }
  \end{center}
\end{sidewaystable}


\subsection{Ordenação por Reversões e Transposições}
\label{subsec:analise_r_t}
Os modelos de ordenação por reversões e transposições,
tabela \ref{table:r_t} obtiveram os piores resultados\footnote{Nenhum
modelo conseguiu resolver instâncias com permutações de tamanho $n >
10$, dentro do tempo limite de $25$ horas.}. Isto já era esperado,
devido ao fato que os modelos utilizam tanto a operação de reversão
como a operação de transposição, resultando em um espaço de busca
maior e no aumento do tempo necessário para encontrar a solução.

O modelo baseado na teoria CSP que usa o
limitante \textit{r\_t\_cc\_csp} desenvolvido para o \textit{ECLiPSe}
não conseguiu resolver a instância com permutações de tamanho $n = 7$
devido ao limite de memória do sistema. Foi o único modelo de
ordenação por reversões e transposições que ocorreu este problema.

É possível notar claramente o rápido crescimento do espaço de busca
dos modelos conforme o tamanho das permutações. Como exemplo, podemos
pegar os modelos baseados na teoria CSP desenvolvidos para
o \textit{ILOG CP}. O modelo que obteve o pior tempo para a instância
com permutações de tamanho $n = 10$ foi o modelo \textit{def\_csp},
que não utiliza limitantes, e o seu tempo de execução para essa
instância foi de $0.012$ segundos. Um tempo excelente para o modelo
que, diferentemente dos modelos que utilizam as operações
isoladamente, conseguiu solucionar esta instância. Porém para as
instâncias com permutações de tamanho $n > 10$, os modelos não
conseguiram solucionar todas permutações dentro do limite de tempo.

Nenhum modelo de ordenação por transposições conseguiu solucionar as
instâncias com permutações de tamanho $n > 10$ dentro do tempo limite
de $25$ horas.

% DO NOT FORGET THE PACKAGE 'rotating'!!!!
% DO NOT FORGET THE PACKAGE 'array'!!!!
% DO NOT FORGET THE PACKAGE 'multirow'!!!!
\begin{sidewaystable}[!ht]
  \caption[Modelos de Ordenação por Reversões e Transposições.]{Tempo médio (em segundos) para o modelo de ordernação por reversões e transposições. O caractere ``-'' significa que o modelo não conseguiu terminar o conjunto de testes dentro do limite de 25 horas. O caractere ``*'' significa que o modelo não conseguir terminar devido ao limite de memória do sistema.}
  \label{table:r_t}
  \begin{center}
  \scalebox{0.520000}{
    \begin{tabular}{| r | >{\raggedleft\arraybackslash}p{1.8cm} | >{\raggedleft\arraybackslash}p{1.8cm} | >{\raggedleft\arraybackslash}p{1.8cm} | >{\raggedleft\arraybackslash}p{1.8cm} | >{\raggedleft\arraybackslash}p{1.8cm} | >{\raggedleft\arraybackslash}p{1.8cm} | >{\raggedleft\arraybackslash}p{1.8cm} | >{\raggedleft\arraybackslash}p{1.8cm} | >{\raggedleft\arraybackslash}p{1.8cm} | >{\raggedleft\arraybackslash}p{1.8cm} | >{\raggedleft\arraybackslash}p{1.8cm} | >{\raggedleft\arraybackslash}p{1.8cm} | >{\raggedleft\arraybackslash}p{1.8cm} | >{\raggedleft\arraybackslash}p{1.8cm} | >{\raggedleft\arraybackslash}p{1.8cm} | >{\raggedleft\arraybackslash}p{1.8cm} | >{\raggedleft\arraybackslash}p{1.8cm} | >{\raggedleft\arraybackslash}p{1.8cm} |}
      \hline
      \multicolumn{19}{|c|}{\textbf{Modelos de Ordenação por Reversões e Transposições}} \\
      \hline
      \multirow{4}{*}{\textbf{size}} & \multicolumn{16}{c|}{\textbf{CP}} & \multicolumn{2}{c|}{\textbf{ILP}} \\
      \cline{02-19}
        & \multicolumn{8}{c|}{\textbf{ECLiPSe}} & \multicolumn{8}{c|}{\textbf{ILOG CP}} & \centering\multirow{3}{*}{\textbf{GLPK}} & \multirow{3}{1.8cm}{\centering\textbf{ILOG CPLEX}} \\
      \cline{02-17}
        & \multicolumn{4}{c|}{\textbf{CSP}} & \multicolumn{4}{c|}{\textbf{COP}} & \multicolumn{4}{c|}{\textbf{CSP}} & \multicolumn{4}{c|}{\textbf{COP}} &  & \\
      \cline{02-17}
        & \centering\textbf{~def~} & \centering\textbf{~r\_t\_br~} & \centering\textbf{~r\_t\_bc~} & \centering\textbf{~r\_t\_cc~} & \centering\textbf{~def~} & \centering\textbf{~r\_t\_br~} & \centering\textbf{~r\_t\_bc~} & \centering\textbf{~r\_t\_cc~} & \centering\textbf{~def~} & \centering\textbf{~r\_t\_br~} & \centering\textbf{~r\_t\_bc~} & \centering\textbf{~r\_t\_cc~} & \centering\textbf{~def~} & \centering\textbf{~r\_t\_br~} & \centering\textbf{~r\_t\_bc~} & \centering\textbf{~r\_t\_cc~} &  & \\
      \hline
      ~3~ & ~0.035~ & ~0.010~ & ~0.004~ & ~0.006~ & ~0.004~ & ~0.005~ & ~0.004~ & ~0.008~ & ~0.001~ & ~0.001~ & ~0.001~ & ~0.001~ & ~0.001~ & ~0.001~ & ~0.001~ & ~0.001~ & ~0.001~ & ~0.001~ \\
      ~4~ & ~6.434~ & ~10.333~ & ~0.248~ & ~0.687~ & ~0.022~ & ~0.026~ & ~0.035~ & ~0.081~ & ~0.002~ & ~0.002~ & ~0.002~ & ~0.003~ & ~0.001~ & ~0.001~ & ~0.001~ & ~0.001~ & ~0.008~ & ~0.008~ \\
      ~5~ & ~-~ & ~-~ & ~30.824~ & ~64.170~ & ~0.379~ & ~0.414~ & ~0.680~ & ~2.066~ & ~0.019~ & ~0.020~ & ~0.022~ & ~0.023~ & ~0.001~ & ~0.001~ & ~0.001~ & ~0.001~ & ~0.466~ & ~0.059~ \\
      ~6~ & ~-~ & ~-~ & ~519.803~ & ~-~ & ~11.566~ & ~12.776~ & ~21.264~ & ~89.760~ & ~0.340~ & ~0.327~ & ~0.319~ & ~0.319~ & ~0.001~ & ~0.001~ & ~0.001~ & ~0.001~ & ~4.642~ & ~1.374~ \\
      ~7~ & ~-~ & ~-~ & ~-~ & ~-~ & ~400.904~ & ~441.946~ & ~792.422~ & ~*~ & ~2.052~ & ~2.062~ & ~2.066~ & ~2.088~ & ~0.004~ & ~0.004~ & ~0.002~ & ~0.004~ & ~5.094~ & ~82.241~ \\
      ~8~ & ~-~ & ~-~ & ~-~ & ~-~ & ~-~ & ~-~ & ~-~ & ~-~ & ~11.894~ & ~11.727~ & ~12.367~ & ~12.369~ & ~0.002~ & ~0.004~ & ~0.004~ & ~0.004~ & ~-~ & ~82.220~ \\
      ~9~ & ~-~ & ~-~ & ~-~ & ~-~ & ~-~ & ~-~ & ~-~ & ~-~ & ~310.012~ & ~304.331~ & ~331.275~ & ~328.956~ & ~0.006~ & ~0.006~ & ~0.004~ & ~0.005~ & ~-~ & ~-~ \\
      ~10~ & ~-~ & ~-~ & ~-~ & ~-~ & ~-~ & ~-~ & ~-~ & ~-~ & ~-~ & ~-~ & ~-~ & ~-~ & ~0.012~ & ~0.010~ & ~0.010~ & ~0.011~ & ~-~ & ~-~ \\
      \hline
    \end{tabular}
  }
  \end{center}
\end{sidewaystable}


\subsection{Sistemas Proprietários x Livres}
\label{subsec:analise_close_open}
Um dos objetivos deste trabalho é analisar se os softwares
proprietários são inferiores ou superiores aos softwares de código
aberto.

No caso das formulações em programação linear inteira, podemos notar
que as formulações desenvolvidas para o \textit{ILOG CPLEX} obtiveram
os melhores tempos para as instâncias com permutações de tamanho
$n \le 7$, mas para permutações com $n > 7$ as formulações
desenvolvidas para o \textit{GLPK} foram mais rápidas, com exceção da
instância com permutações de tamanho $n = 10$ que o \textit{GLPK} não
conseguiu solucionar dentro do tempo limite, ao contrário
do \textit{ILOG CPLEX}. Então para programação linear inteira podemos
dizer que o \textit{ILOG CPLEX} é o mais adequado para os problemas,
devido ao fato de resolver uma instância a mais em relação
ao \textit{GLPK}.

No caso dos modelos de programação por restrições, podemos notar que
os modelos baseados na teoria COP, o \textit{ILOG CP} foi superior
ao \textit{ECLiPSe}. Mas para os modelos baseados na teoria CSP,
o \textit{ECLiPSe} não só obteve tempos melhores, mas também resolveu
instâncias maiores, exceto nos modelos de ordenação por reversões e
transposições. Então para programação por restrições podemos dizer que
o \textit{ECLiPSe} é o mais adequado.


%%%%%%%% CSP vs COP %%%%%%%%%%%
