% conceitos basicos
Neste capítulo faremos uma apresentação dos conceitos básicos
necessários para o entendimento e desenvolvimento deste trabalho. Na
seção \ref{sec:defin} mostraremos as definições usadas no decorrer
deste trabalho. As seções \ref{sec:trans}, \ref{sec:rev} e
\ref{sec:trans_rev} descrevem, respectivamente, os eventos de
transposição, reversão e transposição e reversão.

\section{Definições}
\label{sec:defin}

Para todos os problemas usamos as seguintes definições.

\textit{Permutação.} Para fins computacionais,
um genoma é representado por uma $n$-tupla de genes, e quando não há
genes repetidos essa $n$-tupla é chamada de permutação. Uma permutação
é representada como $\pi = ( \pi_{1}~\pi_{2}~\cdots~\pi_{n} )$, para
$\pi_{i} \in \mathbb{N}$, $0 < \pi_{i} \leq n$ e $i \neq j
\leftrightarrow \pi_{i} \neq \pi_{j}$. A permutação identidade é
representada como $\iota = (1~2~3~\cdots~n)$.

\textit{Eventos de rearranjo.} Os eventos de rearranjo tratados neste
trabalho são os eventos de transposição e reversão quando ocorrem
isoladamente e quando ocorrem de forma conjunta. Os eventos são
representados por $\rho$ e são aplicados a $\pi$ de uma maneira
específica.

\section{Transposição}
\label{sec:trans}
TODO

\section{Reversão}
\label{sec:rev}
TODO

\section{Transposição e Reversão}
\label{sec:trans_rev}
TODO
