% abstract
The evolutionary process was the main responsible for the
differentiation between living beings and one of contemporary theories
about the way as this process occurs states that, during the course of
evolution, genetic changes occurred, creating different kinds of
living beings. Many of these changes are due to point mutation that
modify the DNA sequence, preventing that information is expressed, or
it is expressed in another way. The sequence comparison is the most
common method to characterize the occurrence of point mutation, and is
one of the most discussed problems in Computational Biology. The
interest in making such comparison is to find the edit
distance~\cite{SetubalMeidanis*1997}. The edit distance is a measure
capable of estimating the evolutionary distance between to sequences,
but lacks the information of which global operations were used to
change one sequence in another. These global operations are called the
Genome Rearrangements, which may be, for example, reversals,
transpositions, fissions and fusions. The distance can be defined for
any rearrangement class as the smallest number of operations required
to change one sequence into another. For example, the reversal
distance is the smallest number of reversals required to change one
sequence into another~\cite{BafnaPevzner*1996} and the transposition
distance is the smallest number of
transpositions~\cite{BafnaPevzner*1998}. This work follows the
research line used by Dias e Dias~\cite{DiasDias*2009} and we
introduce Constraint Logic Programming models for sorting by reversals
and sorting by reversals and transpositions, based on Constraint
Satisfaction Problems theory and Constraint Optimization Problems
theory. We made a comparison between the Constraint Logic Programming
models for sorting by transpositions, described in Dias and
Dias~\cite{DiasDias*2009}, sorting by reversals and sorting by
reversals and transpositions, with the Integer Linear Programming for
the same problems described in Dias and Souza~\cite{DiasSouza*2007}.

