% introducao
O processo evolutivo foi o principal responsável pela diferenciação
entre os seres vivos e uma das teorias contemporâneas acerca do modo
como ocorre esse processo afirma que, durante o curso da evolução,
mudanças genéticas aconteceram, criando diferentes espécies de seres
vivos.

Muitas dessas mudanças são devido a mutações pontuais que alteram a
cadeia de DNA, impedindo que a informação seja expressa, ou que seja
expressa de um modo diferente. Tais alterações debilitam, na maioria
dos casos, o organismo portador ou proporcionam vantagens no processo
de seleção natural.

A comparação de sequência é o método mais usual de se caracterizar a
ocorrência de mutações pontuais, sendo um dos problemas mais abordados
em Biologia Computacional. O interesse em fazer tal comparação é
encontrar a distância de edição~\cite{SetubalMeidanis*1997}, que é o
número mínimo de operações de inserções, remoções e substituições
necessárias para transformar uma sequência em outra.

A distância de edição é uma medida capaz de estimar a distância
evolutiva entre duas cadeias, mas não possui a informação de quais
operações globais foram utilizadas para a transformação de uma
sequência em outra. Estas operações globais são os chamados
Rearranjos de Genomas, que podem ser, por exemplo, reversões,
transposições, fissões e fusões.

Um conceito de distância pode ser definido para qualquer classe de
rearranjo como sendo o menor número de operações pertencentes a essa
classe que são necessárias para transformar um genoma em outro. Por
exemplo, chama-se a distância de reversão o menor número de reversões
necessárias para transformar um genoma em
outro~\cite{BafnaPevzner*1996} e a distância de transposição é o menor
número de transposições~\cite{BafnaPevzner*1998}.

Estudos mostram que os rearranjos de genomas são mais apropriados que
mutações pontuais quando se deseja comparar os genoma de duas
espécies~\cite{PalmerHerbon*1988}. Nesse contexto, a distância
evolutiva entre dois genomas pode ser estimada pelo conceito de
distância para uma classe de rearranjo definido no parágrafo anterior.

Neste trabalho, trataremos os casos em que os eventos de transposição e
reversão ocorrem de forma isolada e os casos quando os dois eventos
ocorrem ao mesmo tempo.

% transposição
O evento de transposição ocorre quando dois blocos adjacentes no
genoma trocam de posição. O problema da distância de transposição é
encontrar o número mínimo de transposições necessárias para
transformar um genoma em outro. Este problema pertence à classe dos
problemas NP-Difíceis e a prova foi apresentada por Bulteau, Fertin e
Rusu~\cite{BulteauFertinRusu*2010}. O melhor algoritmo de aproximação
conhecido possui razão de $1.375$ e foi apresentado por Elias e
Hartman~\cite{EliasHartman*2006}.

% reversão
O evento de reversão ocorre quando um bloco do genoma é invertido. O
problema da distância de reversão é encontrar o número mínimo de
reversões necessárias para transformar um genoma em outro. Neste
problema é importante saber se a orientação dos genes é conhecida,
pois existem algoritmos polinomiais para este caso. Entretanto, se a
orientação dos genes não é conhecida o problema da distância de
reversão pertence à classe de problemas NP-Difíceis, com a prova
apresentada por Caprara~\cite{Caprara*1997}. Neste caso, o melhor
algoritmo de aproximação conhecido possui razão de $1.375$ apresentado
por Berman, Hannenhalli e
Karpinski~\cite{BermanHannenhalliKarpinski*2002}.

% reversão+transposição
Na natureza um genoma não sofre apenas eventos de reversão ou de
transposição, ele está exposto a diversos eventos diferentes. Para
esta situação, iremos estudar o caso onde os eventos de reversão e
transposição ocorrem simultaneamente sobre um genoma. Os trabalhos de
Walter, Dias e
Meidanis~\cite{MeidanisWalterDias*2002,WalterDiasMeidanis*1998} e Lin
e Xue~\cite{LinXue*1999} estudaram o problema de encontrar o número
mínimo de reversões e transposições necessárias para transformar um
genoma em outro.

O trabalho desenvolvido nesta dissertação segue a linha de pesquisa
utilizada por Dias e Dias~\cite{DiasDias*2009} e nós apresentaremos
modelos de Programação por Restrições (CP\footnote{Do
inglês \estr{Constraint Programming}.}) para ordenação por reversões e
ordenação por reversões e transposições, baseados na teoria do
Problema de Satisfação de Restrições (CSP\footnote{Do
inglês \estr{Constraint Satisfaction Problems}.}) e na teoria do
Problema de Otimização com Restrições (COP\footnote{Do
inglês \estr{Constraint Optimization Problems}.}). Nós fizemos
comparações com os modelos de Programação por Restrições para
ordenação por transposições, descrito por Dias e
Dias~\cite{DiasDias*2009}, e no caso de ordenação por reversões e
ordenação por reversões e transposições, com os modelos de Programação
Linear Inteira (ILP\footnote{Do inglês \estr{Integer Linear
Programming}.}) descritos por Dias e Souza~\cite{DiasSouza*2007}.

O texto da dissertação está dividido da seguinte maneira. O
Capítulo \ref{cap:basic} apresenta uma série de conceitos básicos
necessários para o entendimento deste trabalho. O
Capítulo \ref{cap:model} descreve os modelos usados neste trabalho. O
Capítulo \ref{cap:resul} traz a análise dos resultados obtidos durante
o trabalho. O Capítulo \ref{cap:concl} apresenta as conclusões da
dissertação.
