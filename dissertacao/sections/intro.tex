% introdução
A teoria da seleção natural de Darwin afirma que os seres vivos atuais
descendem de ancestrais, e ao longo da evolução, mudanças genéticas
propiciaram o aparecimento de diferentes espécies de seres vivos. 

Muitas mutações são pontuais, alterando a cadeia de DNA, o que pode
impedir que a informação seja expressa, ou pode expressá-la de um modo
diferente. Tais alterações debilitam, na maioria dos casos, o organismo
portador ou proporcionam vantagens no processo de seleção natural.

A comparação de sequências é o método mais usual de se identificar a
ocorrência de mutações pontuais, sendo um dos problemas mais abordados
em Biologia Computacional. Um método de comparar duas sequências é
encontrar a distância de edição~\cite{SetubalMeidanis*1997}, que é o
número mínimo de operações de inserções, remoções e substituições
necessárias para transformar uma sequência em outra. 

A distância de edição é uma medida capaz de estimar a distância
evolutiva entre duas cadeias, mas não possui a informação de quais
operações ocorreram para a transformação de uma sequência em outra.

Outra abordagem usada é a de Rearranjo de Genomas, que tem como objetivo
encontrar o menor número de operações que transformam um cromossomo em
outro. Essas operações podem ser, por exemplo, reversões, onde um bloco
do cromossomo é invertido, transposições, onde dois blocos adjacentes no
cromossomo trocam de posição, fissões, efetua a quebra do cromossomo em
dois cromossomos, e fusões, junta dois cromossomos em um único
cromossomo.

O conceito de distância de rearranjo pode ser definido para estas
operações, por exemplo, a distância de reversão é o número mínimo de
reversões que transformam um genoma~\footnote{Quando usamos genoma,
estamos referindo a um determinado cromossomo.} em
outro~\cite{BafnaPevzner*1996} e a distância de transposição é o número
mínimo de transposições que transformam um genoma em
outro~\cite{BafnaPevzner*1998}. 

Estudos mostram que os rearranjos de genomas são mais apropriados que
mutações pontuais quando se deseja comparar genoma de certas
espécies~\cite{PalmerHerbon*1988}, por exemplo nas espécies de plantas
\textit{Brassica}, e no DNA cloroplasto das espécies \textit{Tobacco
fervens} e \textit{Lobelia fervens}~\cite{BafnaPevzner*1996}. Nesse
contexto, a distância evolutiva entre dois genomas pode ser estimada
pelo conceito de distância para uma ou mais operações de rearranjo.

Neste trabalho, trataremos os casos em que os eventos de transposição e
reversão ocorrem de forma isolada e os casos quando os dois eventos
ocorrem ao mesmo tempo.

% Os problemas
% reversão
A operação de reversão ocorre quando um bloco do genoma é invertido. O
problema da distância de reversão é encontrar o número mínimo de
reversões necessárias para transformar um genoma em outro. Neste
problema é importante saber se a orientação dos genes é conhecida,
pois existem algoritmos polinomiais para este caso. Entretanto, se a
orientação dos genes não é conhecida o problema da distância de
reversão pertence à classe de problemas NP-Difíceis, com a prova
apresentada por Caprara~\cite{Caprara*1997}. Neste caso, o melhor
algoritmo de aproximação conhecido possui razão de $1.375$ apresentado
por Berman, Hannenhalli e
Karpinski~\cite{BermanHannenhalliKarpinski*2002}.

% transposição
A operação de transposição ocorre quando dois blocos adjacentes no
genoma trocam de posição. O problema da distância de transposição é
encontrar o número mínimo de transposições necessárias para
transformar um genoma em outro. Este problema pertence à classe dos
problemas NP-Difíceis e a prova foi apresentada por Bulteau, Fertin e
Rusu~\cite{BulteauFertinRusu*2010}. O melhor algoritmo de aproximação
conhecido possui razão de $1.375$ e foi apresentado por Elias e
Hartman~\cite{EliasHartman*2006}.

% reversão+transposição
Na natureza um genoma não sofre apenas eventos de reversão ou de
transposição isoladamente, ele pode sofrer mutações causadas por
operações mutacionais diferentes. Para esta situação, iremos estudar o
caso onde as operações de reversão e transposição ocorrem
simultaneamente sobre um genoma. Os trabalhos de Walter, Dias e
Meidanis~\cite{MeidanisWalterDias*2002,WalterDiasMeidanis*1998} e Lin e
Xue~\cite{LinXue*1999} estudaram o problema de encontrar o número mínimo
de reversões e transposições necessárias para transformar um genoma em
outro.

% O método usado
Nós criamos modelos de Programação por Restrições para ordenação por
reversões e ordenação por reversões e transposições, seguindo a linha de
pesquisa utilizada por Dias e Dias~\cite{DiasDias*2009}. Nós
apresentaremos os modelos de Programação por Restrições (CP\footnote{Do
inglês \estr{Constraint Programming}.}) que buscam os resultados exatos
para os problemas de ordenação por reversões, ordenação por
transposições e ordenação por reversões e transposições, baseados na
teoria do Problema de Satisfação de Restrições (CSP\footnote{Do inglês
\estr{Constraint Satisfaction Problems}.}) e na teoria do Problema de
Otimização com Restrições (COP\footnote{Do inglês \estr{Constraint
Optimization Problems}.}). O modelo de programação por restrições para o
problema de ordenação por reversões e transposições foi apresentado no
artigo ``\textit{Constraint Logic Programming Models for Reversal and
Transposition Distance Problems}''~\cite{IizukaDias*2011}, publicado no
\textit{VI Brazilian Symposium on Bioinformatics (BSB'2011)}.

% Objetivos
Nós fizemos comparações com os modelos de Programação por Restrições
para ordenação por transposições, descrito por Dias e
Dias~\cite{DiasDias*2009}, e com as formulações de Programação Linear
Inteira (ILP\footnote{Do inglês \estr{Integer Linear Programming}.}) que
buscam os resultados exatos para os problemas de ordenação por
reversões, ordenação por transposições e ordenação por reversões e
transposições, descritos por Dias e Souza~\cite{DiasSouza*2007}.
Tanto os modelos de CP quanto as formulações de ILP foram escritas
usando softwares proprietários e softwares de código aberto, com o
objetivo de comparar seus desempenhos, verificando se os softwares
proprietários são inferiores ou superiores aos software de código
aberto.

O texto da dissertação está dividido da seguinte maneira. O
Capítulo \ref{cap:basic} apresenta conceitos básicos
necessários para o entendimento deste trabalho. O
Capítulo \ref{cap:model} descreve os modelos usados neste trabalho. O
Capítulo \ref{cap:resul} traz a análise dos resultados obtidos durante
o trabalho. O Capítulo \ref{cap:concl} apresenta as conclusões da
dissertação.

