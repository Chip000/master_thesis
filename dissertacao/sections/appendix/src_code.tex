%%% Codigo Fonte

\lstset{language=Prolog, 
  numbers=left, 
  numberstyle=\footnotesize\color{gray},
  basicstyle=\ttfamily,
  commentstyle=\color{blue},
  frame=leftline, 
  breaklines=true, 
  breakatwhitespace=false,
  extendedchars=true,
  showlines=true,
  inputencoding=utf8, 
  literate={á}{{\'a}}1
           {ã}{{\~a}}1
           {â}{{\^a}}1
           {à}{{\`a}}1
           {é}{{\'e}}1
           {ê}{{\^e}}1
           {í}{{\'i}}1
           {õ}{{\~o}}1
           {ú}{{\'u}}1
           {ç}{{\c{c}}}1}

Este apêndice apresentará os códigos fontes utilizados nesta dissertação.

\section{ECLiPSe}
\label{app:src_eclipse}
Nesta seção apresentaremos o código fonte desenvolvido para o
sistema \textit{ECLiPSe}. Dividimos por partes para simplificar o
entendimento dos modelos.

O início do arquivo~(\ref{src:basic}) contém predicados simples que
facilitam a codificação dos predicados de ordenação.

\vspace{1.0cm}

\lstinputlisting[
  linerange=1-83,
  caption={Predicados básicos},
  label=src:basic]{src/lib_proj.clp}

Os predicados a seguir~(\ref{src:bounds}) são usados para calcular os
limitantes, usando as ferramentas para ordenação por reversões e
ordenação por transposições citadas no Capítulo~\ref{cap:basic}.

\vspace{1.0cm}

\lstinputlisting[
  linerange=84-486,
  firstnumber=last,
  caption={Calculando os limitantes},
  label=src:bounds]{src/lib_proj.clp}

Os próximos predicados~(\ref{src:perm}) são usados para representar
  permutações e permutações estendidas.

\vspace{1.0cm}

\lstinputlisting[
  linerange=487-503,
  firstnumber=last,
  caption={Permutações},
  label=src:perm]{src/lib_proj.clp}

O predicado \textit{bound}, no próximo trecho de código, escolhe qual
limitante será usado e retorna os valores dos limitantes.

\vspace{1.0cm}

\lstinputlisting[
  linerange=504-550,
  firstnumber=last,
  caption={Predicado bound},
  label=src:bound]{src/lib_proj.clp}

No trecho de código a seguir~(\ref{src:csps}), apresentaremos os
predicados dos problemas de ordenações usando a teoria CSP. Note que é
possível modificar o código para obter quais permutações foram utilizadas
para ordenar a permutação original.

\vspace{1.0cm}

\lstinputlisting[
  linerange=551-630,
  firstnumber=last,
  caption={Modelos CSPs},
  label=src:csps]{src/lib_proj.clp}

O último trecho~(\ref{src:cops}), modela os problemas de ordenações
usando a teoria COP. 

\vspace{1.0cm}

\lstinputlisting[
  firstline=631,
  firstnumber=last,
  caption={Modelos COPs},
  label=src:cops]{src/lib_proj.clp}

