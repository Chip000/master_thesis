% abstract
The darwin's natural selection theory states that living beings of
nowadays are descended from ancestors, and through evolution, genetic
changes led to the appearance of different kinds of living beings. Many
mutations are point mutations, modifying the DNA sequence, which may
prevent the information from being expressed, or may express it in
another way. The sequence comparison is the most common method to
identify the occurrence of point mutations, and is one of the most
discussed problems in Computational Biology. Genome Rearrangement aims
to find the minimum number of operations required to change one sequence
into another. These operations may be, for example, reversals,
transpositions, fissions and fusions. The concept of distance may be
defined for these events, for example, the reversal distance is the
minimum number of reversals required to change one sequence into
another~\cite{BafnaPevzner*1996} and the transposition distance is the
minimum number of transpositions required to change one sequence into
another~\cite{BafnaPevzner*1998}. We will deal with the cases in which
reversals and transpositions events occur separately and the cases in
which both events occur simultaneously, aiming to find the exact value
for the distance. We have created Constraint Programming models for
sorting by reversals and sorting by reversals and transpositions,
following the research line used by Dias and Dias~\cite{DiasDias*2009}.
We will present Constraint Logic Programming models for sorting by
reversals, sorting by transpositions and sorting by reversals and
transpositions, based on Constraint Satisfaction Problems theory and
Constraint Optimization Problems theory. We made a comparison between
the Constraint Logic Programming models for sorting by transpositions,
described in Dias and Dias~\cite{DiasDias*2009}, and with the Integer
Linear Programming formulations for sorting by reversals, sorting by
transpositions and sorting by reversals and transpositions, described in
Dias and Souza~\cite{DiasSouza*2007}.

