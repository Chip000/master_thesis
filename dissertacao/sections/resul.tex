% resultados
Neste capítulo apresentaremos os resultados obtidos pelos modelos
descritos no Capítulo~\ref{cap:model}. A Seção~\ref{sec:tspec} mostra as
características do computador utilizado para executar os testes. A
Seção~\ref{sec:testes} descreve como os testes foram executados. A
Seção~\ref{sec:analise} apresenta a análise dos resultados obtidos
durante este trabalho.

\section{Especificações Técnicas}
\label{sec:tspec}
O computador\footnote{Computador disponível com as licenças dos
softwares proprietários.} utilizado para executar os testes possui as
seguintes características:

\begin{itemize}

    \item{Processador: Intel\textregistered{}~Core\texttrademark~2 Duo
        2.33GHz.}

    \item{Memória RAM: 3 GB.}

    \item{Sistema Operacional: Ubuntu Linux com kernel 2.6.31.}

    \item{Compilador para linguagem C++: gcc 4.4.3~\cite{gcc*2012}.}

\end{itemize}

Todos os modelos de \pr{} foram implementados usando as seguintes
ferramentas:

\begin{itemize}
    \item{Sistema de programação de código aberto
        \textit{ECLiPSe-6.0}~\cite{eclipse*2009}, usando a linguagem
        própria, baseada no prolog.}

    \item{O código foi escrito na linguagem C++, usando o pacote
        proprietário \textit{IBM\textregistered{} ILOG\textregistered{}
        CPLEX\textregistered{} CP Optimizer v 2.3}\footnote{Quando
        usamos ILOG CP, estamos falando sobre este pacote de programação
        por restrições.}~\cite{ilogcp*2011}.}
\end{itemize}

Todas as formulações de programação linear inteira foram implementadas
usando as seguintes ferramentas:

\begin{itemize}
    \item{Sistema de programação de código
        aberto \textit{GLPK-4.35}~\cite{glpk*2010}, usando a linguagem
        de modelagem \textit{GNU MathProg}.}

    \item{O código foi escrito na linguagem C++, usando o pacote
        proprietário \textit{IBM\textregistered{} ILOG\textregistered{}
        CPLEX\textregistered{} Optimizer v 12.1}\footnote{Quando usamos
        ILOG CPLEX, estamos falando sobre este pacote de programação
        linear.}~\cite{ilogcplex*2011}.}
\end{itemize}

\section{Descrição dos Testes}
\label{sec:testes}
Os testes foram separados de acordo com o tamanho das permutações. Uma
instância contém um conjunto de permutações com tamanho $n$, onde $n >
2$ devido ao fato de ser trivial ordenar uma permutação com tamanho $2$.
Para cada instância, geramos $50$ permutações aleatórias com tamanho
$n$.

Todas as instâncias foram executadas nos softwares indicados na
Seção~\ref{sec:tspec}. Para cada instância foi dado o tempo máximo de
$25$ horas, decidido após alguns testes iniciais, onde foi possível
observar que havia um número pequeno de instâncias que não eram
solucionadas devido ao limite de memória do sistema. Os modelos de
programação por restrições e as formulações de programação linear
inteira produzem o resultado exato para a distância escolhida. Fazemos a
comparação dos modelos baseando nos tempos médios usados para resolver
cada instância. Como referência usamos os modelos de \pli{} descritos na
Seção~\ref{sec:pli}.

\section{Análise dos Resultados}
\label{sec:analise}
As tabelas~\ref{table:rev},~\ref{table:trans},~\ref{table:r_t}
apresentam os tempos médios usados para resolver cada instância dos
testes. O caractere ``-'' significa que o modelo não conseguiu
solucionar todas as permutações da instância dentro do limite de 25
horas. O caractere ``*'' significa que o modelo não conseguiu terminar
devido ao limite de memória do sistema.

Podemos observar nos três casos que os modelos de programação por
restrições baseados na teoria COP possuem os piores tempos de execução e
os modelos baseados na teoria CSP possuem os melhores resultados.

O modelo baseado na teoria COP tem como objetivo otimizar o resultado do
problema. Seu mecanismo de busca consiste em encontrar uma solução base
para depois encontrar uma solução melhor, usando um valor melhor para a
função de custo. Com isso, ele acaba gerando um espaço de busca maior do
que o modelo correspondente baseado na teoria CSP, que usa uma
estratégia \textit{bottom-up}.

Também podemos notar que, quanto melhor os limitantes, menor é o tempo
necessário para solucionar as instâncias, conseguindo resolver mais
instâncias com permutações ``maiores''\footnote{Nenhum modelo conseguiu
resolver instâncias com permutações de tamanho $n > 14$, dentro do
tempo limite de $25$ horas.}. Isto ocorre pela redução do conjunto das
possíveis soluções do problema.

A seguir faremos a análise separadamente para cada caso. A
Seção~\ref{subsec:analise_rev} contém os resultados para o problema de
ordenação por reversões. A Seção~\ref{subsec:analise_tra} contém os
resultados para o problema de ordenação por transposições. A
Seção~\ref{subsec:analise_r_t} contém os resultados para o problema de
ordenação por reversões e transposições. A
Seção~\ref{subsec:analise_ferramentas} apresenta a comparação das
ferramentas utilizadas nos testes.

\subsection{Ordenação por Reversões}
\label{subsec:analise_rev}
Nos modelos de ordenação por reversões, Tabela~\ref{table:rev}, podemos
notar que alguns modelos não conseguiram solucionar as permutações
devido ao limite de memória do sistema. Isto ocorreu com as permutações
de tamanho $n = 10$ usando os três limitantes nos modelos baseados na
teoria CSP desenvolvido para o \textit{ILOG CP}, com as permutações de
tamanho $n = 13$ para o limitante \textit{rev\_cg\_csp} no modelo
baseado na teoria CSP e com as permutações de tamanho $n = 6$ para o
limitante \textit{rev\_cg\_cop} no modelo baseado na teoria COP
desenvolvidos para o \textit{ECLiPSe}.

No \textit{ECLiPSe}, é possível observar que nos modelos baseados na
teoria COP, quanto melhor o limitante, maior é o tempo necessário para
resolver as instâncias. O principal motivo é a complexidade existente
para encontrar os melhores limitantes, juntamente com o aumento do
espaço de busca gerado pelo modelo COP. 

A diferença na complexidade pode ser observada quando comparamos os
resultados de ordenação por reversões com os resultados de ordenação por
transposições, Tabela~\ref{table:trans}. É possível notar que, nos
resultados dos limitantes triviais\footnote{Limitante inferior igual à
$0$ e limitante superior igual ao tamanho da permutação fornecida como
entrada.}, colunas \textit{def\_cop} e \textit{def\_csp}, e dos
limitantes no grafo de \textit{breakpoints} para reversões, colunas
\textit{rev\_br\_cop} e \textit{rev\_br\_csp} e dos limitantes que usam
\textit{breakpoints} para transposições, colunas \textit{tra\_br\_cop} e
\textit{tra\_br\_csp}, os tempos dos modelos de ordenação por reversões
é melhor, em relação aos modelos de transposições. Isto ocorre, devido à
procura dos blocos na permutação, que irá sofrer o evento escolhido. A
reversão irá alterar apenas um bloco, ou seja, necessita apenas escolher
duas posições na permutação. No caso da transposição, o evento trocará
dois blocos adjacentes de lugar, necessitando escolher três posições na
permutação. Logo o espaço de busca dos modelos de ordenação por
transposição é maior do que o espaço dos modelos de ordenação por
reversões.

Entretanto, se focarmos apenas nos melhores limitantes dos dois tipos de
ordenação, colunas \textit{rev\_cg\_cop}, \textit{rev\_cg\_csp},
\textit{tra\_cg\_cop} e \textit{tra\_cg\_csp}, ocorre justamente o
inverso. Este comportamento pode ser explicado usando a decomposição em
ciclos. Na transposição a decomposição é única, para todo vértice do
grafo de ciclos, toda aresta chegando é unicamente pareada com uma
aresta saindo de cor diferente, então basta encontrar o número de ciclos
ímpares. No caso da reversão, todo vértice possui o mesmo número de
arestas incidentes cinzas e pretas no grafo de \bkp{}. Logo existem
diversas maneiras para realizar a decomposição em ciclos. Como o modelo
usa os limitantes definidos por Christie~\cite{Christie*1998}, é
necessário mais processamento para encontrar a decomposição máxima em
$2$-ciclos, aumentando o espaço de busca dos modelos de ordenação por
reversões.

A diferença na complexidade pode ser notado, também, nos modelos
baseados na teoria CSP desenvolvidos para o \textit{ILOG CP}. Isto pode
ser causado por dois motivos:
\begin{enumerate}

    \item{Forma da modelagem: Como o modelo foi pensado para o
        \textit{ECLiPSe}, o modelo não aproveita características
        específicas do \textit{ILOG CP} que poderiam melhorar sua
        performance.}

    \item{Mecanismo de \textit{backtracking}\footnote{Algoritmo para
        encontrar soluções para um problema computacional. Ele pode
        eliminar múltiplas soluções apenas se decidir que elas não são
        viáveis para o problema.}: O \textit{ECLiPSe} usa uma linguagem 
        baseada no Prolog, que possui o paradigma de programação 
        lógica. Uma das suas características é ter o mecanismo de 
        \textit{backtracking} embutido na linguagem. No caso do 
        \textit{ILOG CP}, o modelo foi escrito usando a linguagem C++, 
        que é uma linguagem orientada à objetos e não tem o mecanismo 
        de \textit{backtracking} por padrão. Portanto, o 
        \textit{ECLiPSe} realiza o \textit{backtracking} de forma 
        ``mais natural'' do que o \textit{ILOG CP}.}

\end{enumerate}

Nenhum modelo de ordenação por reversões conseguiu solucionar as
instâncias com permutações de tamanho $n > 13$ dentro do tempo limite de
$25$ horas.

% DO NOT FORGET THE PACKAGE 'rotating'!!!!
\begin{sidewaystable}[!ht]
  \caption{Tempo médio (em segundos) para o modelo de ordernação por reversões. O caractere ``-'' significa que o modelo não conseguiu terminar o conjunto de testes dentro do limite de 25 horas.}
  \label{table:rev}
  \begin{center}
    \begin{tabular}{| r | r | r | r | r | r | r | r | r | r | r |}
      \hline
      \multicolumn{11}{|c|}{\textbf{Reversals Models}} \\
      \hline
      \textbf{size} & \multicolumn{8}{|c|}{\textbf{CP}} & \multicolumn{2}{|c|}{\textbf{ILP}} \\
      \cline{02-11}
        & \multicolumn{4}{|c|}{\textbf{ECLiPSe}} & \multicolumn{4}{|c|}{\textbf{ILOG CP}} & \textbf{GLPK} & \textbf{ILOG CPLEX}  \\
      \cline{02-09}
        & \textbf{~def\_cop~} & \textbf{~rev\_br\_cop~} & \textbf{~def\_csp~} & \textbf{~rev\_br\_csp~} & \textbf{~def\_cop~} & \textbf{~rev\_br\_cop~} & \textbf{~def\_csp~} & \textbf{~rev\_br\_csp~}  & & \\
      \hline
      ~3~ & ~0.009~ & ~0.004~ & ~0.003~ & ~0.004~ & ~0.002~ & ~0.001~ & ~0.004~ & ~0.003~ & ~0.001~ & ~0.002~ \\
      ~4~ & ~0.417~ & ~0.409~ & ~0.027~ & ~0.005~ & ~0.006~ & ~0.006~ & ~0.008~ & ~0.004~ & ~0.001~ & ~0.008~ \\
      ~5~ & ~35.502~ & ~43.286~ & ~0.280~ & ~0.010~ & ~0.018~ & ~0.017~ & ~0.016~ & ~0.011~ & ~0.096~ & ~0.036~ \\
      ~6~ & ~-~ & ~-~ & ~5.223~ & ~0.026~ & ~0.095~ & ~0.096~ & ~0.063~ & ~0.037~ & ~1.264~ & ~0.562~ \\
      ~7~ & ~-~ & ~-~ & ~490.753~ & ~0.226~ & ~1.494~ & ~1.356~ & ~0.334~ & ~0.261~ & ~4.702~ & ~16.011~ \\
      ~8~ & ~-~ & ~-~ & ~-~ & ~1.096~ & ~20.217~ & ~29.556~ & ~4.360~ & ~4.164~ & ~4.428~ & ~426.984~ \\
      ~9~ & ~-~ & ~-~ & ~-~ & ~6.885~ & ~989.500~ & ~1458.167~ & ~217.353~ & ~216.878~ & ~-~ & ~-~ \\
      ~10~ & ~-~ & ~-~ & ~-~ & ~30.742~ & ~-~ & ~-~ & ~-~ & ~-~ & ~-~ & ~-~ \\
      \hline
    \end{tabular}
  \end{center}
\end{sidewaystable}


\subsection{Ordenação por Transposições}
\label{subsec:analise_tra}
Nos modelos de ordenação por transposições, Tabela~\ref{table:trans}, os
modelos baseados na teoria COP que usam o limitante
\textit{tra\_br\_cop} apresentaram os piores resultados, não apresentando
nenhuma vantagem em relação ao modelo que usa limitantes triviais
\textit{def\_cop}.

No \textit{ECLiPSe}, é possível observar que, diferentemente do modelo
de ordenação por reversões, nos modelos baseados na teoria COP, o melhor
limitante, \textit{tra\_cg\_cop}, obteve o melhor tempo de execução.
Este limitante conseguiu reduzir o espaço de busca por ser mais preciso
que os outros. Neste caso, a redução do espaço de busca supriu a
necessidade da quantidade de processamento para encontrar o número de
ciclos ímpares no grafo de ciclos da ordenação por transposições, ao
contrário do modelo de ordenação por reversões que usa o limitante
\textit{rev\_cg\_cop}. Em relação aos modelos baseados na teoria CSP, o
modelo de ordenação por transposições que usa o limitante
\textit{tra\_cg\_csp} obteve tempos de execução melhores que o modelo de
ordenação por reversões que usa o limitante \textit{rev\_cg\_csp}.

Similar ao modelo de ordenação por reversões, nos modelos baseados na
teoria CSP desenvolvidos para o \textit{ILOG CP}, quanto melhor o
limitante, maior é o tempo necessário para resolver as instâncias.

Nenhum modelo de ordenação por transposições conseguiu solucionar as
instâncias com permutações de tamanho $n > 14$ dentro do tempo limite de
$25$ horas.

% DO NOT FORGET THE PACKAGE 'rotating'!!!!
\begin{sidewaystable}[!ht]
  \caption{Tempo médio (em segundos) para o modelo de ordernação por transposições. O caractere ``-'' significa que o modelo não conseguiu terminar o conjunto de testes dentro do limite de 25 horas.}
  \label{table:trans}
  \begin{center}
    \scalebox{0.7}{
    \begin{tabular}{| r | r | r | r | r | r | r | r | r | r | r | r | r | r | r |}
      \hline
      \multicolumn{15}{|c|}{\textbf{Transpositions Models}} \\
      \hline
      \textbf{size} & \multicolumn{12}{|c|}{\textbf{CP}} & \multicolumn{2}{|c|}{\textbf{ILP}} \\
      \cline{02-15}
        & \multicolumn{6}{|c|}{\textbf{ECLiPSe}} & \multicolumn{6}{|c|}{\textbf{ILOG CP}} & \textbf{GLPK} & \textbf{ILOG CPLEX}  \\
      \cline{02-13}
        & \textbf{~def\_cop~} & \textbf{~tra\_br\_cop~} & \textbf{~tra\_cg\_cop~} & \textbf{~def\_csp~} & \textbf{~tra\_br\_csp~} & \textbf{~tra\_cg\_csp~} & \textbf{~def\_cop~} & \textbf{~tra\_br\_cop~} & \textbf{~tra\_cg\_cop~} & \textbf{~def\_csp~} & \textbf{~tra\_br\_csp~} & \textbf{~tra\_cg\_csp~}  & & \\
      \hline
      ~3~ & ~0.005~ & ~0.008~ & ~0.003~ & ~0.003~ & ~0.003~ & ~0.001~ & ~0.003~ & ~0.003~ & ~0.001~ & ~0.004~ & ~0.001~ & ~0.002~ & ~0.001~ & ~0.001~ \\
      ~4~ & ~0.553~ & ~1.704~ & ~0.066~ & ~0.021~ & ~0.007~ & ~0.001~ & ~0.006~ & ~0.005~ & ~0.005~ & ~0.008~ & ~0.006~ & ~0.004~ & ~0.001~ & ~0.011~ \\
      ~5~ & ~198.019~ & ~1479.668~ & ~3.276~ & ~0.335~ & ~0.023~ & ~0.004~ & ~0.041~ & ~0.014~ & ~0.010~ & ~0.029~ & ~0.021~ & ~0.011~ & ~0.196~ & ~0.057~ \\
      ~6~ & ~-~ & ~-~ & ~38.020~ & ~14.197~ & ~0.156~ & ~0.008~ & ~0.243~ & ~0.074~ & ~0.045~ & ~0.104~ & ~0.075~ & ~0.043~ & ~2.348~ & ~1.375~ \\
      ~7~ & ~-~ & ~-~ & ~-~ & ~920.502~ & ~1.416~ & ~0.029~ & ~1.588~ & ~1.253~ & ~0.429~ & ~0.559~ & ~0.472~ & ~0.266~ & ~4.650~ & ~79.015~ \\
      ~8~ & ~-~ & ~-~ & ~-~ & ~-~ & ~12.544~ & ~0.076~ & ~23.778~ & ~26.807~ & ~16.619~ & ~4.803~ & ~4.537~ & ~3.798~ & ~-~ & ~-~ \\
      ~9~ & ~-~ & ~-~ & ~-~ & ~-~ & ~49.813~ & ~0.382~ & ~1109.692~ & ~994.209~ & ~385.954~ & ~112.203~ & ~111.656~ & ~66.439~ & ~-~ & ~-~ \\
      ~10~ & ~-~ & ~-~ & ~-~ & ~-~ & ~1287.331~ & ~2.297~ & ~-~ & ~-~ & ~-~ & ~-~ & ~-~ & ~-~ & ~-~ & ~-~ \\
      \hline
    \end{tabular}
}
  \end{center}
\end{sidewaystable}


\subsection{Ordenação por Reversões e Transposições}
\label{subsec:analise_r_t}
Os modelos de ordenação por reversões e transposições,
Tabela~\ref{table:r_t}, obtiveram os piores resultados. Isto já era
esperado, devido ao fato que os modelos utilizam tanto a operação de
reversão como a operação de transposição, resultando em um espaço de
busca maior e no aumento do tempo necessário para encontrar a solução.

O modelo baseado na teoria CSP que usa o limitante
\textit{r\_t\_cc\_csp} desenvolvido para o \textit{ECLiPSe} não
conseguiu resolver a instância com permutações de tamanho $n = 7$ devido
ao limite de memória do sistema. Foi o único modelo de ordenação por
reversões e transposições que ocorreu este problema.

É possível notar claramente o rápido crescimento do espaço de busca dos
modelos conforme o tamanho das permutações. Como exemplo, podemos pegar
os modelos baseados na teoria CSP desenvolvidos para o \textit{ILOG CP}.
O modelo que obteve o pior tempo para a instância com permutações de
tamanho $n = 10$ foi o modelo \textit{def\_csp}, que não utiliza
limitantes, sendo o tempo de execução para essa instância $0.012$
segundos. Um tempo excelente para o modelo que, diferentemente dos
modelos que utilizam as operações isoladamente, conseguiu solucionar
esta instância. Porém para as instâncias com permutações de tamanho $n >
10$, os modelos não conseguiram solucionar todas permutações dentro do
limite de tempo.

Nenhum modelo de ordenação por reversões e transposições conseguiu
solucionar as instâncias com permutações de tamanho $n > 10$ dentro do
tempo limite de $25$ horas.

% DO NOT FORGET THE PACKAGE 'rotating'!!!!
\begin{sidewaystable}[!ht]
  \caption{Tempo médio (em segundos) para o modelo de ordernação por reversões e transposições. O caractere ``-'' significa que o modelo não conseguiu terminar o conjunto de testes dentro do limite de 25 horas.}
  \label{table:r_t}
  \begin{center}
    \scalebox{0.7}{
    \begin{tabular}{| r | r | r | r | r | r | r | r | r | r | r | r | r | r | r |}
      \hline
      \multicolumn{15}{|c|}{\textbf{Reversals and Transpositions Models}} \\
      \hline
      \textbf{size} & \multicolumn{12}{|c|}{\textbf{CP}} & \multicolumn{2}{|c|}{\textbf{ILP}} \\
      \cline{02-15}
        & \multicolumn{6}{|c|}{\textbf{ECLiPSe}} & \multicolumn{6}{|c|}{\textbf{ILOG CP}} & \textbf{GLPK} & \textbf{ILOG CPLEX}  \\
      \cline{02-13}
        & \textbf{~def\_cop~} & \textbf{~r\_t\_br\_cop~} & \textbf{~r\_t\_bc\_cop~} & \textbf{~def\_csp~} & \textbf{~r\_t\_br\_csp~} & \textbf{~r\_t\_bc\_csp~} & \textbf{~def\_cop~} & \textbf{~r\_t\_br\_cop~} & \textbf{~r\_t\_bc\_cop~} & \textbf{~def\_csp~} & \textbf{~r\_t\_br\_csp~} & \textbf{~r\_t\_bc\_csp~}  & & \\
      \hline
      ~3~ & ~0.034~ & ~0.012~ & ~0.003~ & ~0.004~ & ~0.003~ & ~0.002~ & ~0.004~ & ~0.002~ & ~0.001~ & ~0.004~ & ~0.002~ & ~0.003~ & ~0.001~ & ~0.002~ \\
      ~4~ & ~7.370~ & ~12.288~ & ~0.341~ & ~0.028~ & ~0.007~ & ~0.004~ & ~0.008~ & ~0.008~ & ~0.007~ & ~0.012~ & ~0.009~ & ~0.005~ & ~0.001~ & ~0.012~ \\
      ~5~ & ~-~ & ~-~ & ~26.047~ & ~0.343~ & ~0.020~ & ~0.010~ & ~0.026~ & ~0.024~ & ~0.021~ & ~0.031~ & ~0.021~ & ~0.013~ & ~0.396~ & ~0.055~ \\
      ~6~ & ~-~ & ~-~ & ~409.079~ & ~16.742~ & ~0.122~ & ~0.031~ & ~0.268~ & ~0.232~ & ~0.103~ & ~0.085~ & ~0.066~ & ~0.046~ & ~4.062~ & ~0.808~ \\
      ~7~ & ~-~ & ~-~ & ~-~ & ~593.666~ & ~0.670~ & ~0.104~ & ~1.896~ & ~1.967~ & ~1.179~ & ~0.533~ & ~0.400~ & ~0.255~ & ~3.660~ & ~94.429~ \\
      ~8~ & ~-~ & ~-~ & ~-~ & ~-~ & ~2.579~ & ~0.149~ & ~12.851~ & ~10.589~ & ~5.566~ & ~3.088~ & ~2.655~ & ~1.531~ & ~-~ & ~-~ \\
      ~9~ & ~-~ & ~-~ & ~-~ & ~-~ & ~13.958~ & ~0.339~ & ~468.581~ & ~422.396~ & ~102.687~ & ~61.973~ & ~60.465~ & ~19.984~ & ~-~ & ~-~ \\
      ~10~ & ~-~ & ~-~ & ~-~ & ~-~ & ~64.208~ & ~1.318~ & ~-~ & ~-~ & ~-~ & ~-~ & ~-~ & ~1189.290~ & ~-~ & ~-~ \\
      \hline
    \end{tabular}
}
  \end{center}
\end{sidewaystable}


\subsection{Comparação das ferramentas}
\label{subsec:analise_ferramentas}
Um dos objetivos deste trabalho é analisar se os softwares proprietários
são inferiores ou superiores aos softwares de código aberto. Usaremos a
Tabela~\ref{table:n_inst} para analisar o número de instâncias
resolvidas por cada ferramenta utilizada.

No caso das formulações em programação linear inteira, podemos notar que
as formulações desenvolvidas para o \textit{ILOG CPLEX} obtiveram os
melhores tempos para as instâncias com permutações de tamanho $n < 7$, e
as formulações desenvolvidas para o \textit{GLPK} foram mais rápidas nas
permutações de tamanho $n = 7$, para todas formulações, e $n = 8$, no
caso de ordenação por reversões, ver Tabela~\ref{table:rev}, mas não
conseguiu resolver instâncias com permutações de tamanho $n > 8$, ao
contrário do \textit{ILOG CPLEX}.  Se analisamos somente o número de
instâncias resolvidas, ver Tabela~\ref{table:n_inst}, podemos afirmar
que \textit{ILOG CPLEX} é o mais adequado para os problemas de
ordenação, devido ao fato de resolver uma instância a mais em relação ao
\textit{GLPK}. Apesar do pequeno número de instâncias resolvidas pelas
formulações de programação linear inteira, podemos notar que escrever as
formulações usando o pacote do \textit{ILOG CPLEX} é mais adequado para
os três problemas de ordenação.

No caso dos modelos de programação por restrições, iremos dividir a
análise em dois casos. Nos modelos baseados na teoria COP, podemos notar
que, nos três casos, o \textit{ILOG CP} foi superior ao
\textit{ECLiPSe}, tanto nos tempos de execução, quanto no número de
instâncias resolvidas, sendo $70$ instâncias resolvidas pelo
\textit{ILOG CP} e $30$ instâncias resolvidas pelo \textit{ECLiPSe}.
Então, para os modelos baseados na teria COP, podemos afirmar que o
pacote do \textit{ILOG CP} é o mais adequado para os problemas de
ordenação.

Mas para os modelos baseados na teoria CSP, o \textit{ECLiPSe} não só
obteve tempos melhores, mas também resolveu instâncias maiores. No caso
do problema de ordenação por reversões, o \textit{ECLiPSe} resolveu $24$
instâncias com permutações de tamanho até $n~=~12$, enquanto que o
\textit{ILOG CP} resolveu $21$ instâncias com permutações de tamanho até
$n = 9$. No caso do problema de ordenação por transposições, o
\textit{ECLiPSe} resolveu $24$ instâncias com permutações de tamanho até
$n = 14$, enquanto que o \textit{ILOG CP} resolveu $21$ instâncias com
permutações de tamanho até $n = 9$. Mas no caso do problema de ordenação
por reversões e transposições, o \textit{ILOG CP} foi melhor, resolvendo
$32$ instâncias com permutações de tamanho até $n = 10$, enquanto o
\textit{ECLiPSe} resolveu $19$ instâncias com permutações de tamanho até
$n = 7$.  Então, para o problema de ordenação por reversões e
transposições, o \textit{ILOG CP} é o mais adequado, mas para os outros
problemas, podemos afirmar que o \textit{ECLiPSe} é o mais adequado.

\pagebreak
\clearpage
\newpage

\begin{table}[ht]
    \caption[Número de instâncias resolvidas]{Quantidade de instâncias
    resolvidas por cada ferramenta utilizada.}
    \label{table:n_inst}

    \begin{center}
        \begin{tabular}{|>{\raggedright}p{3.0cm}| 
                         >{\raggedleft\arraybackslash}p{1.3cm}|
                         >{\raggedleft\arraybackslash}p{1.3cm}|
                         >{\raggedleft\arraybackslash}p{1.3cm}|
                         >{\raggedleft\arraybackslash}p{1.3cm}|
                         >{\raggedleft\arraybackslash}p{1.4cm}|
                         >{\raggedleft\arraybackslash}p{1.5cm}|
                         >{\raggedleft\arraybackslash}p{1.3cm}|}
                     
            \hline
            
            \multicolumn{8}{|c|}{\textbf{Número de instâncias
            Resolvidas}} \\
            
            \hline
            
            & %\multirow{2}{*}{\textbf{Ordenação}} & 
            \multicolumn{2}{c|}{\textbf{ECLiPSe}} &
            \multicolumn{2}{c|}{\textbf{ILOG CP}} &
            \multirow{2}{1.4cm}{\centering\textbf{GLPK}} & 
            \multirow{2}{1.5cm}{\centering\textbf{ILOG CPLEX}} &
            \multirow{2}{1.3cm}{\centering\textbf{Total}} \\

            \cline{02-05}

            & \centering\textbf{COP} & \centering\textbf{CSP} &
            \centering\textbf{COP} & \centering\textbf{CSP} & & & \\
            
            \hline

            \textbf{Reversões} & 9 & 24 & 21 & 21 & 6 & 7 & 88 \\

            \hline

            \textbf{Transposições} & 10 & 24 & 21 & 21 & 5 & 5 & 86 \\

            \hline

            \multirow{2}{3.0cm}{\textbf{Reversões e Transposições}} & 
            \multirow{2}{*}{11} & \multirow{2}{*}{19} &
            \multirow{2}{*}{28} & \multirow{2}{*}{32} &
            \multirow{2}{*}{5} &  \multirow{2}{*}{5} &
            \multirow{2}{*}{100} \\

            & & & & & & & \\

            \hline

            \textbf{Total} & 30 & 67 & 70 & 74 & 16 & 17 & 274 \\
            
            \hline

        \end{tabular}
    \end{center}
\end{table}


