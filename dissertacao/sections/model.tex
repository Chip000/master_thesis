% modelos
Neste capítulo nós apresentaremos a descrição dos modelos de \pr{} e
\pli{} usados para os problemas de ordenação por transposições,
ordenação por reversões e ordenação por reversões e transposições.

\section{\PR}
\label{sec:cp}
O modelo de \pr{} usado para o problema de ordenação por transposições
é o descrito em Dias e Dias~\cite{DiasDias*2009}. Nós usamos as
definições contidas no capítulo~\ref{cap:basic} para escrever as
formulações baseadas nas teorias de Problema de Satisfação de
Restrições (CSP) e Problema de Otimização com Restrições (COP). As
formulações foram descritas usando a notação \textit{prolog-like} de
Marriot \cite{Marriott*1998}. Primeiramente iremos apresentar os
predicados que são comum às duas formulações.

Em prolog as variáveis são descritas por \textit{strings} iniciadas
com letra maiúscula ou ``\_'' (\textit{underscore}) caso a variável
seja anônima. As letras gregas $\pi$ e $\sigma$ representam listas
nesta notação. A construção $X~::~[i~..~j]$ significa que $X$ (ou cada
elemento de $X$ se $X$ for uma lista) pode assumir um valor do
intervalo $[i~..~j]$.

A representação da permutação (\ref{perm}) e o efeito das operações de
reversão (\ref{reversal}) e transposição (\ref{transposition} podem
ser vistas da mesma maneira que são descritas pelos problemas. Neste
modelo a permutação $\pi$ é uma lista de elementos
($\pi_{1},~\pi_{2},~\ldots~,~\pi_{n}$) onde $\pi_{i} \in \mathbb{N}$,
$0 < \pi_{i} \le n$ e $\pi_{i} \neq \pi_{j}$ para $i \neq j$.
\begin{align}
  \label{perm}
  \begin{split}
  \textit{per}&\textit{mutation}(\pi, N)~\text{:-} \\
  &\textit{length}(\pi, N), \\ 
  &\pi~::~[1~..~N], \\
  &\textit{all\_different}(\pi). 
  \end{split}
\end{align}

Na reversão $\rho(i,j)$, $0 < i < j \leq n$, dividimos a lista em três
sublistas $C_{1}C_{2}C_{3}$ onde $C_{1} = (\pi_{1}~..~\pi_{i-1})$,
$C_{2} = (\pi_{i}~..~\pi_{j})$ e $C_{3} =
(\pi_{j+1}~..~\pi_{n})$. Depois fazemos a reversão na sublista
$C_{2}$, resultando na lista $R_{C_{2}}$. Então juntamos a nova lista
$R_{C_{2}}$ com as sublistas $C_{1}$ e $C_{3}$ para formar $\rho\pi =
C_{1}R_{C_{2}}C_{3}$.
\begin{align}
  \label{reversal}
  \begin{split}
  \textit{rev}&\textit{ersal}(\pi, \sigma, I, J)~\text{:-} \\
  &\textit{permutation}(\pi, N), \\
  &\textit{permutation}(\sigma, N),  \\
  &1 \le I < J \le N, \\
  &\textit{split}(\pi, I, J, C_{1}, C_{2}, C_{3}), \\
  &\textit{reverse}(C_{2}, R_{C_{2}}),  \\
  &\sigma = C_{1}, R_{C_{2}}, C_{3}. 
  \end{split}
\end{align}

Na transposição $\rho(i,j,k)$, $0 < i < j < k\leq n$, dividimos a
lista em quatro sublistas $C_{1}C_{2}C_{3}C_{4}$ onde $C_{1} =
(\pi_{1}~..~\pi_{i-1})$, $C_{2} = (\pi_{i}~..~\pi_{j-1})$, $C_{3} =
(\pi_{j}~..~\pi_{k-1})$ e $C_{4} = (\pi_{k}~..~\pi_{n})$. Trocamos de
posição os blocos $C_{2}$ e $C_{3}$ e juntamos elas na ordem $C_{1}$,
$C_{3}$, $C_{2}$ e $C_{4}$ para formar $\rho\pi =
C_{1}C_{3}C_{2}C_{4}$. Observe que as sublistas $C_{1}$ e $C_{4}$
podem ser vazias.
\begin{align}
  \label{transposition}
  \begin{split}
  \textit{tra}&\textit{nsposition}(\pi, \sigma, I, J, K)~\text{:-} \\
  &\textit{permutation}(\pi, N), \\
  &\textit{permutation}(\sigma, N), \\
  &1 \le I < J < K \le N,  \\
  &\textit{split}(\pi, I, J, K, C_{1}, C_{2}, C_{3}, C_{4}), \\
  &\sigma = C_{1}, C_{3}, C_{2}, C_{4}. 
  \end{split}
\end{align}

\subsection{Modelo CSP}
\label{subsec:modelcsp}
Primeiramente modelaremos o problema usando a teoria CSP, mas o número
de variáveis é desconhecido devido ao fato de precisarmos do valor da
distância de reversão $d_{r}(\pi)$ para criar as restrições e
variáveis que representam as permutações. Por esta razão, nós
escolhemos um valor candidato para a distância $R$ tal que $R \in
[LB~..~UB]$, onde $LB$ é um limitante inferior conhecido e $UB$ é um
limitante superior conhecido para o problema, e tentamos achar uma
combinação apropriada de $R$ reversões que solucionam o problema. Se o
modelo CSP falha (não existe combinação que soluciona o problema com o
valor candidato escolhido) com o candidato $R$, nós escolhemos outro
valor $R$ apenas incrementando seu valor. O valor de $R$ é escolhido
usando uma estratégia \textit{bottom-up}, ou seja, a verificação
inicia pelo valor do limitante inferior $LB$ e termina quando o valor
é maior que o limitante superior $UB$. Na transposição, o processo é o
mesmo que na reversão, trocando apenas o valor da distância de
reversão ($d_{r}(\pi)$) para o valor da distância de transposição
($d_{t}(\pi)$).
\begin{align}
  \label{revdistance}
  \begin{split}
  \textit{rev}&\textit{ersal\_distance}(\iota, 0, \_Model). \\
  \textit{rev}&\textit{ersal\_distance}(\pi, R, Model)~\text{:-} \\
  &\textit{bound}(\pi, Model, LB, UB), \\
  &R :: [LB~..~UB], \\
  &\textit{indomain}(R),  \\
  &\textit{reversal}(\pi, \sigma, \_I, \_J),  \\
  &\textit{reversal\_distance}(\sigma, R-1, Model). 
  \end{split}
\end{align}
\begin{align}
  \label{tradistance}
  \begin{split}
  \textit{tra}&\textit{nsposition\_distance}(\iota, 0, \_Model). \\
  \textit{tra}&\textit{nsposition\_distance}(\pi, T, Model)~\text{:-} \\
  &\textit{bound}(\pi, Model, LB, UB), \\
  &T :: [LB~..~UB], \\
  &\textit{indomain}(T),  \\
  &\textit{transposition}(\pi, \sigma, \_I, \_J, \_K),  \\
  &\textit{transposition\_distance}(\sigma, T-1, Model). 
  \end{split}
\end{align}

O predicado \textit{rev\_trans\_dist/3} (\ref{trarevdist}) retorna o
valor da distância de reversão e transposição. O
predicado \textit{event/2} escolhe o melhor evento entre o
predicado \textit{reversal/4} (\ref{reversal}) e o
predicado \textit{transposition/5} (\ref{transposition}) para
minimizar o valor da distância.
\begin{align}
  \label{trarevdist}
  \begin{split}
  \textit{rev}&\textit{\_trans\_dist}(\iota, 0, \_Model). \\
  \textit{rev}&\textit{\_trans\_dist}(\pi, N, Model)~\text{:-} \\
  &\textit{bound}(\pi, Model, LB, UB), \\
  &N :: [LB~..~UB], \\
  &\textit{indomain}(N),  \\
  &\textit{event}(\pi, \sigma),  \\
  &\textit{rev\_trans\_dist}(\sigma, N-1, Model). 
  \end{split}
\end{align}

O predicado \textit{indomain(X)} em (\ref{revdistance}),
(\ref{tradistance}) e (\ref{trarevdist}) pega o domínio da variável
$X$ e escolhe o menor elemento dele (no caso o valor do limitante
inferior). Se o modelo retorna para o predicado \textit{indomain}
devido a uma falha, o elemento que originou ela será removido do
domínio e um outro valor será escolhido.

Os modelos CSP para os problemas de ordenação possuem a estrutura
mostrada acima, trocando apenas os limitantes usados. A seguir
apresentaremos os predicados que lidam com a escolha do modelo a ser
usado.

Em comum para os todos eventos descritos temos o
modelo \textit{def\_csp} que não usa nenhum limitante.
\begin{align}
  \label{bound_def}
  \begin{split}
  \textit{bou}&\textit{nd}(\pi, def\_csp, LB, UB)~\text{:-} \\
  &LB = 0, \\
  &\textit{length}(\pi, UB).
  \end{split}
\end{align}

\textit{Ordenação por reversões:}
\begin{itemize}
\item{\textit{rev\_br\_csp}: 
Modelo que usa o conceito de \bkp{} em reversões para calcular os
limitantes descritos no Teorema~\ref{teo:rev_bkp_bound}.}
\item{\textit{rev\_cg\_csp}:
Modelo que usa o número de $2$-ciclos na máxima decomposição em ciclos
de $G(\pi)$ para calcular os limitantes descritos no
Teorema~\ref{teo:rev_cic_bound}.}
\end{itemize}
\begin{align}
  \label{bound_rev}
  \begin{split}
  \textit{bou}&\textit{nd}(\pi, rev\_br\_csp, LB, UB)~\text{:-}  \\
  &\textit{calc\_breakpoints\_reversal}(\pi, B), \\
  &LB = B / 2  \\ % / 2 ,  \\
  &UB = B.  \\
  \textit{bou}&\textit{nd}(\pi, rev\_cg\_csp, LB, UB)~\text{:-}  \\
  &\textit{find\_2\_cycle}(\pi, B, C), \\
  &LB = (2 * B - C) / 3 ,   \\
  &UB = B - C / 2.
  \end{split}
\end{align}

\textit{Ordenação por transposições:}
\begin{itemize}
\item{\textit{tra\_br\_csp}: 
Modelo que usa o conceito de \bkp{} em transposições para calcular os
limitantes conforme descrito no Teorema~\ref{teo:trans_br_bound}.}
\item{\textit{tra\_cg\_csp}:
Modelo que usa o conceito de grafo de ciclos em transposições, fazendo
a decomposição de ciclos e analisando os ciclos ímpares separadamente
para calcular os limitantes conforme descrito no
Teorema~\ref{teo:trans_cg_bound}.}
\end{itemize}
\begin{align}
  \label{bound_tra}
  \begin{split}
  \textit{bou}&\textit{nd}(\pi, tra\_br\_csp, LB, UB)~\text{:-} \\
  &\textit{calc\_breakpoints\_transposition}(\pi, B), \\
  &LB = B / 3  \\ %/ 3,  \\
  &UB = B.   \\
  \textit{bou}&\textit{nd}(\pi, tra\_cg\_csp, LB, UB)~\text{:-} \\
  &\textit{calc\_oddcycle\_transposition}(\pi, N, C), \\
  &LB = (N + 1 - C) / 2  \\ 
  &UB = (3 * (N + 1 - C)) / 4.
  \end{split}
\end{align}

\textit{Ordenação por reversões e transposições:} 
\begin{itemize}
\item{\textit{r\_t\_br\_csp}: 
Melhor limitante superior entre o limitante de \bkp{} para reversões e
o limitante de \bkp{} para transposições.}
\item{\textit{r\_t\_bc\_csp}:
Melhor limitante superior entre o limitante de \bkp{} para reversões e
o limitante do grafo de ciclos para transposições.}
\item{\textit{r\_t\_cc\_csp}: 
Melhor limitante superior entre o limitante do grafo de ciclos para
reversões e o limitante do grafo de ciclos para transposições.}
\end{itemize}
\begin{align}
  \label{bound_r_t}
  \begin{split}
  \textit{bou}&\textit{nd}(\pi, r\_t\_br\_csp, 0, UB)~\text{:-} \\
  &\textit{bound}(\pi, rev\_br, \_RLB, RUB), \\
  &\textit{bound}(\pi, tra\_br, \_TLB, TUB), \\
  &\textit{min}(RUB, TUB, UB). \\
  \textit{bou}&\textit{nd}(\pi, r\_t\_bc\_csp, 0, UB)~\text{:-} \\
  &\textit{bound}(\pi, rev\_br, \_RLB, RUB), \\
  &\textit{bound}(\pi, tra\_cg, \_TLB, TUB), \\
  &\textit{min}(RUB, TUB, UB). \\
  \textit{bou}&\textit{nd}(\pi, r\_t\_cc\_csp, 0, UB)~\text{:-} \\
  &\textit{bound}(\pi, rev\_cg, \_RLB, RUB), \\
  &\textit{bound}(\pi, tra\_cg, \_TLB, TUB), \\
  &\textit{min}(RUB, TUB, UB).
  \end{split}
\end{align}

O predicado \textit{bound/4} (\ref{bound_def}), (\ref{bound_rev}),
(\ref{bound_tra}), (\ref{bound_r_t}) recebe na variável \textit{Model}
um átomo que representa o modelo a ser usado. Este átomo se conecta
com o predicado que retorna o limitante superior e inferior apropriado
para o modelo. Observe que o limitante inferior é igual a $0$ no caso
dos modelos de ordenação por reversões e transposições. Isto ocorre
devido ao fato que a cada nova iteração do modelo pode surgir um
limitante inferior melhor, simplesmente fazendo a troca entre as
operações de reversão e transposição.
%
%\begin{align}
%  \label{bound}
%  \begin{split}
%  \textit{bou}&\textit{nd}(\pi, def\_csp, LB, UB)~\text{:-} \\
%  &\textit{def\_csp\_bound}(\pi, LB, UB).  \\
%  \textit{bou}&\textit{nd}(\pi, rev\_br\_csp, LB, UB)~\text{:-}  \\
%  &\textit{calc\_breakpoints\_reversal}(\pi, B), \\
%  &LB = B / 2  \\ % / 2 ,  \\
%  &UB = B.  \\
%  \textit{bou}&\textit{nd}(\pi, rev\_cg\_csp, LB, UB)~\text{:-}  \\
%  &\textit{find\_2\_cycle}(\pi, B, C), \\
%  &LB = (2 * B - C) / 3 ,   \\
%  &UB = B - C / 2.  \\
%  \textit{bou}&\textit{nd}(\pi, tra\_br\_csp, LB, UB)~\text{:-} \\
%  &\textit{calc\_breakpoints\_transposition}(\pi, B), \\
%  &LB = B / 3  \\ %/ 3,  \\
%  &UB = B.   \\
%  \textit{bou}&\textit{nd}(\pi, tra\_cg\_csp, LB, UB)~\text{:-} \\
%  &\textit{calc\_oddcycle\_transposition}(\pi, N, C), \\
%  &LB = (N + 1 - C) / 2  \\ 
%  &UB = (3 * (N + 1 - C)) / 4.  \\
%  \textit{bou}&\textit{nd}(\pi, r\_t\_br\_csp, 0, UB)~\text{:-} \\
%  &\textit{bound}(\pi, rev\_br, \_RLB, RUB), \\
%  &\textit{bound}(\pi, tra\_br, \_TLB, TUB), \\
%  &\textit{min}(RUB, TUB, UB), \\
%  \textit{bou}&\textit{nd}(\pi, r\_t\_bc\_csp, 0, UB)~\text{:-} \\
%  &\textit{bound}(\pi, rev\_br, \_RLB, RUB), \\
%  &\textit{bound}(\pi, tra\_cg, \_TLB, TUB), \\
%  &\textit{min}(RUB, TUB, UB), \\
%  \textit{bou}&\textit{nd}(\pi, r\_t\_cc\_csp, 0, UB)~\text{:-} \\
%  &\textit{bound}(\pi, rev\_cg, \_RLB, RUB), \\
%  &\textit{bound}(\pi, tra\_cg, \_TLB, TUB), \\
%  &\textit{min}(RUB, TUB, UB), 
%  \end{split}
%\end{align}

\subsection{Modelo COP}
\label{subsec:modelcop}
Uma outra alternativa é modelar o problema usando a teoria COP. Os
modelos que usam esta abordagem necessitam de um limitante superior,
portanto serão feitas algumas alterações nos predicados definidos
anteriormente. Nós usamos as variáveis binárias $B$ para indicar
quando uma operação de reversão ou de transposição modificou ou não a
permutação fornecida.

O primeiro predicado que precisamos criar para o evento de reversão é
o \textit{reversal\_cop/5} (\ref{reversal_cop}). Primeiramente, dado
uma reversão $\rho(i, j)$, adicionamos uma nova restrição para
permitir $(i, j) = (0, 0)$. Se $(i, j) = (0, 0)$ então $\pi\rho
= \pi$. Então, adicionamos um novo argumento ao
predicado \textit{reversal\_cop} que recebe a variável $B$.
\begin{align}
  \label{reversal_cop}
  \begin{split}
  \textit{rev}&\textit{ersal\_cop}(\iota, \iota, 0, 0, 0). \\
  \textit{rev}&\textit{ersal\_cop}(\pi, \sigma, I, J, 1)~\text{:-}~ 
  \textit{reversal}(\pi, \sigma, I, J).
  \end{split}
\end{align}

O predicado equivalente para o evento de transposição é
o \textit{transposition\_cop/6} (\ref{transposition_cop}). Neste caso,
dado uma transposição $\rho(i, j, k)$, adicionamos uma nova restrição
para permitir $(i, j, k) = (0, 0, 0)$. Se $(i, j, k) = (0, 0, 0)$
então $\pi\rho = \pi$.
\begin{align}
  \label{transposition_cop}
  \begin{split}
  \textit{tra}&\textit{nsposition\_cop}(\iota, \iota, 0, 0, 0, 0). \\
  \textit{tra}&\textit{nsposition\_cop}(\pi, \sigma, I, J, K, 1)~\text{:-}~ 
  \textit{transposition}(\pi, \sigma, I, J, K). 
  \end{split}
\end{align}

Para calcular a distância de reversão nos modelos baseados na teoria
COP, implementamos o predicado \textit{reversal\_distance\_cop/3}
(\ref{revdistance_cop}), que ajusta as variáveis $B$ usando o valor do
limitante superior e restringe as permutações fazendo $\pi_{k}
= \pi_{k-1} \rho_{k}$. O predicado \textit{length/2}, predicado
interno do prolog, é usado para criar uma lista de variáveis não
instanciadas com o tamanho dado. A função de custo \textit{Cost} é a
soma das variáveis $B$ associadas com cada $\rho_{k}$, $Cost
= \sum_{k=1}^{UB} B_{k}$, onde $UB$ é um limitante superior
conhecido. A distância de reversão é o valor mínimo da função de custo
$d_{r} = \min Cost$. Para evitar processamento desnecessários, o valor
de $Cost$ precisa ser maior ou igual a qualquer limitante inferior. O
predicado equivalente para o problema de ordenação por transposições é
o \textit{transposition\_distance\_cop/3} (\ref{tradistance_cop}).
\begin{align}
  \label{revdistance_cop}
  \begin{split}
  \textit{rev}&\textit{ersal\_distance\_cop}(\pi, R, Model)~\text{:-} \\
  &\textit{bound}(\pi, Model, LB, UB), \\
  &\textit{length}(B, UB),  \\
  &\textit{upperbound\_constraint\_rev}(\pi, B, Model, UB), \\
  &\textit{sum}(B, Cost),  \\
  &\textit{Cost} \ge \textit{LB},  \\
  &\textit{minimize}(Cost, R). 
  \end{split}
\end{align}
\begin{align}
  \label{tradistance_cop}
  \begin{split}
  \textit{tra}&\textit{nsposition\_distance\_cop}(\pi, T, Model)~\text{:-} \\
  &\textit{bound}(\pi, Model, LB, UB), \\
  &\textit{length}(B, UB),  \\
  &\textit{upperbound\_constraint\_trans}(\pi, B, Model, UB), \\
  &\textit{sum}(B, Cost),  \\
  &\textit{Cost} \ge \textit{LB},  \\
  &\textit{minimize}(Cost, T). 
  \end{split}
\end{align}

O predicado equivalente para o modelo de ordenação por reversões e
transposições é
o \textit{rev\_trans\_dist\_cop/3}~(\ref{trarevdistcop}). O
predicado \textit{upperbound\_constraint\_event/4} escolhe o melhor
evento entre a reversão, usando o
predicado \textit{upperbound\_constraint\_rev/4}
(\ref{ub_constaint_rev}), e a transposição, usando o
predicado \textit{upperbound\_constraint\_trans/4}
(\ref{ub_constaint_tra}), para minimizar o valor da distância.
\begin{align}
  \label{trarevdistcop}
  \begin{split}
  \textit{rev}&\textit{\_trans\_dist\_cop}(\pi, N, Model)~\text{:-} \\
  &\textit{bound}(\pi, Model, LB, UB), \\
  &\textit{length}(B, UB),  \\
  &\textit{upperbound\_constraint\_event}(\pi, B, Model, UB), \\
  &\textit{sum}(B, Cost),  \\
  &\textit{Cost} \ge \textit{LB},  \\
  &\textit{minimize}(Cost, N). 
  \end{split}
\end{align}

O predicado \textit{upperbound\_constraint\_rev/4}
(\ref{ub_constaint_rev}) aplica na permutação os efeitos de $\rho_{k}$
e retorna o valor apropriado de $B$ para cada reversão $\rho_{k}$. Uma
restrição importante é verificar se é possível ordenar a permutação
usando o número restante de reversões para evitar processamento
desnecessário. O predicado equivalente para o evento de transposição é
o \textit{upperbound\_constraint\_trans/4} (\ref{ub_constaint_tra}).
\begin{align}
  \label{ub_constaint_rev}
  \begin{split}
  \textit{upp}&\textit{erbound\_constraint\_rev}(\iota, [~], \_Model, \_UB). \\
  \textit{upp}&\textit{erbound\_constraint\_rev}(\pi, [B|Bt], Model, UB)~\text{:-} \\
  &\textit{reversal\_cop}(\pi, \sigma, \_I, \_J, B), \\
  &\textit{bound}(\pi, Model, LB, \_UB), \\
  &UB \ge LB,  \\
  &\textit{upperbound\_constraint\_rev}(\sigma, Bt, Model, UB - 1).
  \end{split}
\end{align}
\begin{align}
  \label{ub_constaint_tra}
  \begin{split}
  \textit{upp}&\textit{erbound\_constraint\_trans}(\iota, [~], \_Model, \_UB). \\
  \textit{upp}&\textit{erbound\_constraint\_trans}(\pi, [B|Bt], Model, UB)~\text{:-} \\
  &\textit{transposition\_cop}(\pi, \sigma, \_I, \_J, \_K, B), \\
  &\textit{bound}(\pi, Model, LB, \_UB), \\
  &UB \ge LB,  \\
  &\textit{upperbound\_constraint\_trans}(\sigma, Bt, Model, UB - 1).
  \end{split}
\end{align}

Os modelos baseados na teoria COP possuem a estrutura acima, trocando
apenas os limitantes usados. Os limitantes são os mesmos usados para
os modelos CSP, modificados para os modelos COP. Então temos os
seguintes limitantes: \textit{def\_cop}, \textit{rev\_br\_cop},
\textit{rev\_cg\_cop}, \textit{tra\_br\_cop}, \textit{tra\_cg\_cop}, 
\textit{r\_t\_br\_cop}, \textit{r\_t\_bc\_cop}
e \textit{r\_t\_cc\_cop}.

\section{\PLI}
\label{sec:pli}
A abordagem utilizada para programação linear inteira é a descrita no
trabalho de Dias e de Souza~\cite{DiasSouza*2007}. Neste trabalho,
define-se um modelo para o problema da distância com tamanho
polinomial em relação ao tamanho da permutação fornecida como
entrada. O modelo é específico para os eventos de reversão,
transposição ou reversão e transposição quando ocorrem
simultaneamente.

Primeiramente vamos apresentar as variáveis e restrições que são
comuns para todos os modelos. A ideia é assegurar que só estamos
tratando com permutações válidas.

\textit{Gerando permutações válidas a cada iteração.} 
As variáveis $B_{ijk}$ indicam se a $i$-ésima posição de $\pi$ possui
o valor $j$ depois da $k$-ésima operação ter sido executada, para todo
$1 \le i, j \le n$ e todo $ 0 \le k < n$.

\[ 
B_{ijk} = \left \{ 
\begin{tabular}{ll} 
  1, & se $\pi[i]$ = $j$ depois da $k$-ésima operação \\ 
  0, & caso contrário
\end{tabular} 
\right .
\]

As restrições (\ref{eq:corr1}) e (\ref{eq:corr2}) garantem que a
permutação inicial e a final são corretas. 
\begin{align} 
  B_{i,\pi[i],0} = 1,~\text{para todo $1 \le i \le n$}. \label{eq:corr1} \\
  B_{i,\sigma[i],n-1} = 1,~\text{para todo $1 \le i \le n$}. \label{eq:corr2}
\end{align}

A restrição (\ref{eq:posval}) garante que cada posição de uma
permutação possui exatamente um valor associado a ela. Já a restrição
(\ref{eq:valpos}) garante que todo valor esteja associado a uma
posição de cada permutação.
\begin{align}
  \sum_{j=1}^{n} B_{ijk} = 1,~\text{para todo $1 \le i \le n$, 
  $0 \le k < n$}. \label{eq:posval} \\
  \sum_{i=1}^{n} B_{ijk} = 1,~\text{para todo $1 \le j \le n$, $0 \le
   k < n$}. \label{eq:valpos}
\end{align}

\textit{Distância de reversão.}
Para o problema da distância de reversão definimos os seguintes
conjuntos de variáveis e restrições. As variáveis binárias $r_{abk}$
indicam quando a $k$-ésima operação de reversão afeta o blocos
$\pi[a~\ldots~b]$ de $\pi$, para todo $1 \le a < b \le n$ e todo
$1 \le k < n$.

\[
r_{abk} = \left \{ 
\begin{tabular}{ll} 
  1, & se $\rho_{k} = \rho(a,b)$ \\
  0, & caso contrário
\end{tabular} 
\right .
\]

As variáveis binárias $r_{k}$ são usadas para decidir se a $k$-ésima
operação de reversão modificou a permutação, para todo $ 1 \le k < n$.

\[
r_{k} = \left \{ 
\begin{tabular}{ll} 
 1, & se $\rho_{k} = \rho(x,y)$ e
 $\rho_{k} \rho_{k-1}~\ldots~\rho_{1} \pi \neq \rho_{k-1}~\ldots~\rho_{1} \pi$\\
 0, & caso contrário 
\end{tabular}
\right .
\]

As restrições (\ref{eq:rev1}) e (\ref{eq:rev2}) são necessárias para
identificar as reversões que fazem parte da solução. A restrição
(\ref{eq:rev1}) garante que se a $k$-ésima reversão não alterar a
permutação, nenhuma das reversões seguintes poderá alterar. Já a
restrição (\ref{eq:rev2}) garante que no máximo uma reversão poderá
ser feita por iteração.
\begin{align}
  &r_{k} \le r_{k-1},~\text{para todo $1 \le k <
  n$}. \label{eq:rev1} \\
  \sum_{a=1}^{n-1}\sum_{b=a+1}^{n}
  &r_{abk} \le r_{k},~\text{para todo $1 \le k < n$}. \label{eq:rev2}
\end{align}

As próximas restrições lidam com as modificações na permutação causada
pela reversão a cada iteração da execução. A análise será dividida em
dois casos onde, para cada caso, analisamos cada posição $i$ da
permutação para verificar seu valor após a operação de reversão
$\rho(a,b)$ ser completada.
\begin{enumerate}
\item{$i < a$ ou $i > b$:
A operação de reversão não modifica estas posições.
\begin{align}
  \begin{split}
    \sum_{a=i+1}^{n-1}\sum_{b=a+1}^{n} r_{abk} +
    \sum_{a=1}^{n-1}\sum_{b=a+1}^{i-1} r_{abk} +
    (1 - r_{k}) + B_{i,j,k-1} - B_{ijk} \le 1, \\
    \text{para todo $1 \le i,~j \le n$ e todo $1 \le k < n$}.
  \end{split}
  \label{eq:rev3}
\end{align}}
\item{$a \le i \le b$:
A operação de reversão altera os elementos guardados nestas
posições. Para não ser redundante, a desigualdade precisa ter os dois
primeiros termos com valor $1$. Neste caso, $B_{b+a-i,j,k-1} = 1$,
implicando que o elemento $j$ foi salvo na posição $b + a - i$ antes
da reversão ser movida para a posição $i$ ($B_{ijk} = 1$).
\begin{align}
  \begin{split}
  r_{abk} + B_{b+a-i,j,k-1} - B_{ijk} \le 1, \\
  \text{$1 \le a < b \le n,~a \le i \le b,~1 \le j
    \le n,~1 \le k < n$}.
  \end{split}
  \label{eq:rev4}
\end{align}}
\end{enumerate}

\textit{Distância de transposição.}
Para o problema da distância de transposição, usaremos os seguintes
conjuntos de variáveis e restrições. As variáveis binárias $t_{abck}$
indicam quando a $k$-ésima operação de transposição realiza a troca de
lugares dos blocos $\pi[a~\ldots~b - 1]$ e $\pi[b~\ldots~c - 1]$ da
permutação $\pi$, para todo $1 \le a < b < c \le n + 1$ e todo $ 1 \le
k < n$.

\[ 
  t_{abck} = \left \{ 
  \begin{tabular}{ll} 
  1, & se $\rho_{k} = \rho(a,b,c)$ \\ 
  0, & caso contrário 
  \end{tabular} \right .
\] 

As variáveis binárias $t_{k}$ são usadas para decidir se a $k$-ésima
operação de transposição modificou a permutação, para todo $ 1 \le k <
n$.

\[ 
  t_{k} = \left \{ 
  \begin{tabular}{ll} 
  1, & se $\rho_{k} = \rho(x,y,z)$ e
  $\rho_{k} \rho_{k-1}~\ldots~\rho_{1} \pi \neq \rho_{k-1}~\ldots~\rho_{1} \pi$\\
  0, & caso contrário 
  \end{tabular} \right .
\]

As restrições (\ref{eq:transp1}) e (\ref{eq:transp2}) são necessárias
para identificar as transposições que fazem parte da solução. A
restrição (\ref{eq:transp1}) garante que se a $k$-ésima transposição
não alterar a permutação, nenhuma das transposições seguintes poderá
alterar. Já a restrição (\ref{eq:transp2}) garante que no máximo uma
transposição poderá ser feita por iteração.
\begin{align}
  &t_{k} \le t_{k-1}, \text{para todo $1 \le k <
  n$}. \label{eq:transp1} \\
  \sum_{a=1}^{n-1}\sum_{b=a+1}^{n}\sum_{c=b+1}^{n+1} &t_{abck} \le
  t_{k} , \text{para todo $1 \le k <
  n$}. \label{eq:transp2} 
\end{align}

As próximas restrições refletem a modificações na permutação causada
por uma transposição a cada passo da execução. A análise será dividida em
três casos onde, para cada caso, analisamos cada posição $i$ da
permutação para verificar seu valor após a operação de transposição
$\rho(a,b,c)$ ser completada.
\begin{enumerate}
\item{$i < a$ ou $i \ge c$:
A operação de transposição não altera estas posições.
\begin{align}
  \begin{split} 
    \sum_{a=i+1}^{n-1}\sum_{b=a+1}^{n}\sum_{c=b+1}^{n+1} t_{abck}
    + \sum_{a=1}^{n-1}\sum_{b=a+1}^{n}\sum_{c=b+1}^{i} t_{abck} + (1 -
    t_{k}) + B_{i,j,k-1} - B_{ijk} \le 1, \\
    \text{para todo $1 \le i, j \le n$ e todo $1 \le k <
    n$}.
  \end{split} \label{eq:transp3}
\end{align}}
\item{$a \le i < a + c -b$:
Após a operação de transposição ser completada, estas posições serão
ocupadas pelos elementos que estão nas posições de $b$ a $c - 1$. Para
não ser redundante, esta desigualdade precisa ter os dois primeiros
termos com o valor $1$. Neste caso, temos que $B_{b-a+i,j,k-1} = 1$,
implicando que o elemento $j$ foi salvo na posição $b - a + i$ antes
da transposição ser movida para a posição $i$ ($B_{ijk} = 1$).
\begin{align}
  \begin{split}
  t_{abck} + B_{b-a+i,j,k-1} - B_{ijk} \le 1, \\
  \text{$1 \le a < b < c \le n + 1, a \le i < a + c -b, 1 \le j
  \le n, 1 \le k < n$}.
  \end{split}
  \label{eq:transp4}
\end{align}}
\item{$a + c - b \le i < c$:
Após a operação de transposição ser completada, estas posições serão
ocupadas pelos elementos que estão nas posições de $a$ a $b -
1$. Similar com o caso acima, esta desigualdade é redundante se os
valores dos dois primeiros termos não for igual a $1$. Isto significa
que a $k$-ésima transposição move $B^{k-1}[a~..~b-1]$ para as posições
que precedem a posição $c$. Por definição, $i$ representa uma das
posições que receberão um elemento deste subvetor. Então, temos que
$B^{k}[i] = B^{k-1}[b - c + i]$, para todo $i \in [a + c - b~..~c -
1]$ e os últimos dois termos se anulam.
\begin{align}
  \begin{split}
  t_{abck} + B_{b-c+i,j,k-1} - B_{ijk} \le 1, \\
  \text{$1 \le a < b < c \le n + 1, a + c - b \le i < c, 1 \le j
  \le n, 1 \le k < n$}.  
  \end{split}
  \label{eq:transp5}
\end{align}}
\end{enumerate}

\textit{Distância de reversão e transposição.}
Para o problema da distância de reversão e transposição usaremos todas
as variáveis definidas anteriormente, com a adição das variáveis
binárias $z_{k}$, que é usada para indicar quando uma $k$-ésima
operação, seja ela uma reversão ou uma transposição, realmente
modificou a permutação. Então, para todo $1 \le k < n$, temos que:

\[
z_{k} = \left \{ 
  \begin{tabular}{ll} 
  1, & se $\rho_{k} = \rho(x,y)$ ou $\rho_{k} = \rho(x,y,z)$ e
  $\rho_{k} \rho_{k-1}~\ldots~\rho_{1} \pi \neq \rho_{k-1}~\ldots~\rho_{1} \pi$\\
  0, & caso contrário
  \end{tabular} \right .
\]

Usaremos todas as restrições definidas anteriormente, com exceção das
restrições (\ref{eq:rev1}) e (\ref{eq:transp1}) que serão substituídas
pelas restrições (\ref{eq:rt1}) e (\ref{eq:rt2}). A restrição
(\ref{eq:rt1}) garante que se não ocorreu operações em uma iteração
então não ocorrerá nenhuma operação nas iterações seguintes. A
restrição (\ref{eq:rt2}) garante que no máximo uma operação é
executada a cada iteração.
\begin{align}
  &z_{k} \le z_{k-1}, \text{para todo $1 \le k <
    n$}. \label{eq:rt1} \\ 
  &r_{k} + t_{k} = z_{k} , \text{para todo
    $1 \le k < n$}. \label{eq:rt2}
\end{align}

Precisamos modificar as restrições (\ref{eq:rev3}) e
(\ref{eq:transp3}), substituindo $r_{k}$ e $t_{k}$ por $z_{k}$,
resultando nas restrições (\ref{eq:rt3}) e (\ref{eq:rt4}).
\begin{align}
  \begin{split}
    \sum_{a=i+1}^{n-1}\sum_{b=a+1}^{n} r_{abk} +
    \sum_{a=1}^{n-1}\sum_{b=a+1}^{i-1} r_{abk} +
    (1 - z_{k}) + B_{i,j,k-1} - B_{ijk} \le 1, \\
    \text{para todo $1 \le i,~j \le n$ e todo $1 \le k < n$}.
  \end{split} \label{eq:rt3} \\
  \begin{split} 
    \sum_{a=i+1}^{n-1}\sum_{b=a+1}^{n}\sum_{c=b+1}^{n+1} t_{abck}
    + \sum_{a=1}^{n-1}\sum_{b=a+1}^{n}\sum_{c=b+1}^{i} t_{abck} + (1 -
    z_{k}) + B_{i,j,k-1} - B_{ijk} \le 1, \\
    \text{para todo $1 \le i, j \le n$ e todo $1 \le k <
    n$}.
  \end{split} \label{eq:rt4}
\end{align}

\subsection{Função Objetivo}
\label{subsec:fo}
Considerando as variáveis e restrições descritas anteriormente para
cada um dos três problemas de distâncias, temos a função objetivo
$\omega_{r} = \min \sum_{k=1}^{n-1} r_{k}$, para o problema da
distância de reversão, a função objetivo $\omega_{t}
= \min \sum_{k=1}^{n-1} t_{k}$, para o problema da distância de
transposição, e a função objetivo $\omega_{rt} = \min \sum_{k=1}^{n-1}
z_{k}$, para o problema da distância de reversão e transposição quando
ocorrem simultaneamente.

\subsection{Tamanho do modelo}
\label{subsec:modelsize}
É fácil observar que o modelo descrito possui tamanho polinomial em
relação ao tamanho da permutação fornecida como entrada. A
tabela~\ref{table:modelsize} mostra o tamanho do modelo para os três
problemas de distâncias com relação ao parâmetro $n$ (tamanho da
permutação de entrada).

\begin{table}[h]
\caption{Tamanho dos modelos em relação à $n$.}
\label{table:modelsize}
\center
\begin{tabular}{|l|c|c|}
  \hline
  \textbf{Modelo} & \textbf{Variáveis} & \textbf{Restrições} \\
  \hline
  Distância de Reversão & $O(n^3)$ & $O(n^5)$ \\
  Distância de Transposição & $O(n^4)$ & $O(n^6)$ \\
  Distância de Reversão e Transposição & $O(n^4)$ & $O(n^6)$ \\
  \hline
\end{tabular}
\end{table}
