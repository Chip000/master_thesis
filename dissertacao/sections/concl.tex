% conclusao 
Neste trabalho, nós criamos modelos de Programação por Restrições para
ordenação por reversões e ordenação por reversões e transposições,
seguindo a linha de pesquisa utilizada por Dias e
Dias~\cite{DiasDias*2009}. Nós apresentamos os modelos de Programação
por Restrições que buscam os resultados exatos para os problemas de
ordenação por reversões, ordenação por transposições e ordenação por
reversões e transposições, baseados na teoria do Problema de Satisfação
de Restrições e na teoria do Problema de Otimização com Restrições.

No Capítulo~\ref{cap:basic} apresentamos os conceitos usados nesta
dissertação. Na Seção~\ref{sec:defin} mostramos as formalizações usadas
pelos problemas de rearranjos de genomas. Nas
seções~\ref{sec:rev},~\ref{sec:trans} e~\ref{sec:rev_trans} descrevemos
os problemas de ordenação por reversões, ordenação por transposições e
ordenação por reversões e transposições, respectivamente. Na
Seção~\ref{sec:def_cp} explicamos o conceito de Programação por
Restrições. Os modelos de Programação por Restrições e as formulações de
Programação Linear Inteira foram descritos no Capítulo~\ref{cap:model}.

No Capítulo~\ref{cap:resul} apresentamos os resultados obtidos. Nós
fizemos comparações com os modelos de Programação por Restrições para
ordenação por transposições, descrito por Dias e
Dias~\cite{DiasDias*2009}, e com as formulações de Programação Linear
Inteira que buscam os resultados exatos para os problemas de ordenação
por reversões, ordenação por transposições e ordenação por reversões e
transposições, descritos por Dias e Souza~\cite{DiasSouza*2007}.

Os resultados foram analisados observando os tempos médios usados para
resolver cada instâncias dos testes. Analisamos também qual ferramenta
é a mais adequada para o problema, no caso de programação linear
inteira usamos o \textit{GLPK} e o \textit{ILOG CPLEX}, e no caso de
programação por restrições usamos o \textit{ECLiPSe} e o \textit{ILOG
CP}, sendo que o \textit{GLPK} e o \textit{ECLiPSe} são softwares de
código aberto, e o \textit{ILOG CPLEX} e o \textit{ILOG CP} são
softwares proprietários.

Os resultados mostraram que os modelos de programação por restrições
baseados na teoria CSP obtiveram os melhores tempos em relação às
formulações de programação linear, já os modelos de programação por
restrições baseados na teoria COP obtiveram os piores tempos, no
modelo de ordenação por reversões, e as formulações de programação
linear inteira obtiveram os piores resultados nos outros problemas de
ordenação.

O modelo de programação por restrições para o
problema de ordenação por reversões e transposições foi apresentado no
artigo ``\textit{Constraint Logic Programming Models for Reversal and
Transposition Distance Problems}''~\cite{IizukaDias*2011}, publicado no
\textit{VI Brazilian Symposium on Bioinformatics (BSB'2011)}, realizado
em Brasília, DF em 2011.

Apesar de ser mais uma ferramenta para solucionar os problemas de
ordenação por reversões, ordenação por transposições e ordenação por
reversões e transposições, esta abordagem ainda é inviável na prática.

Para trabalhos futuros, heurísticas podem ser estudadas para melhorar o
desempenho dos modelos de programação por restrições (observe que não
foi utilizada nenhuma heurística na criação dos modelos). As heurísticas
vão desde a escolha de qual permutação deve ser analisada primeiro até a
análise de permutações que possuem alguma característica em comum. Em
ambos os casos, o objetivo é reduzir o espaço de busca do problema.

Podemos melhorar a formulação de programação linear inteira, utilizando
relaxação lagrangeana, ou escrever uma nova formulação sem preocupação
com o seu tamanho para aplicar técnicas como geração de colunas e
\textit{branch-and-cut}~\cite{NemhauserWolsey*1988,Wolsey*1998}.
