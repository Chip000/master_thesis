% conclusao
Neste trabalho foram apresentados modelos de Programação por
Restrições para ordenação por reversões e ordenação por reversões e
transposições, baseados na teoria do Problema de Satisfação de
Restrições e na teoria do Problema de Otimização com Restrições,
seguindo a linha de pesquisa utilizada por Dias e
Dias~\cite{DiasDias*2009}.

Fizemos a comparação dos modelos de programação por restrições para os
problemas de ordenação por transposições, descrito por Dias e
Dias~\cite{DiasDias*2009}, ordenação por reversões e ordenação por
reversões e transposições com as formulações de programação linear
inteira para os mesmos problemas descritos em Dias e
Souza~\cite{DiasSouza*2007}.

Os resultados foram analisados observando os tempos médios usados para
resolver cada instâncias dos testes. Analisamos também qual ferramenta
é a mais adequada para o problema, no caso de programação linear
inteira usamos o \textit{GLPK} e o \textit{ILOG CPLEX}, e no caso de
programação por restrições usamos o \textit{ECLiPSe} e o \textit{ILOG
CP}.

Os resultados mostraram que os modelos de programação por restrições
baseados na teoria CSP obtiveram os melhores tempos em relação às
formulações de programação linear, já os modelos de programação por
restrições baseados na teoria COP obtiveram os piores tempos, no
modelo de ordenação por reversões, e as formulações de programação
linear inteira obtiveram os piores resultados nos outros problemas de
ordenação.

Apesar de ser mais uma ferramenta para solucionar os problemas de
ordenação por reversões, ordenação por transposições e ordenação por
reversões e transposições, esta abordagem ainda é inviável na
prática.

Um artigo~\cite{IizukaDias*2011} com os principais resultados dessa
dissertação foi apresentado no \textit{VI Brazilian Symposium on
Bioinformatics (BSB'2011)}, realizado em Brasília, DF em 2011.

Para trabalhos futuros, heurísticas podem ser estudadas para melhorar
o desempenho dos modelos de programação por restrições (observe que
não foi utilizada nenhuma heurística na criação dos modelos). As
heurísticas vão desde a escolha de qual permutação deve ser analisada
primeiro, por exemplo o uso de \textit{x-movimento} descrito em Bafna
e Pevzner~\cite{BafnaPevzner*1998} no caso de ordenação por
transposições, até a análise de permutações que possuem alguma
característica em comum, por exemplo o estudo do diâmetro de
transposição para \textit{x-permutações} descrito em Elias e
Hartman~\cite{EliasHartman*2006}.

Podemos melhorar a formulação de programação linear inteira,
utilizando relaxação lagrangeana, ou escrever uma nova formulação sem
se preocupar com o seu tamanho para aplicar técnicas como geração de
colunas
e \textit{branch-and-cut}~\cite{NemhauserWolsey*1988,Wolsey*1998}.
