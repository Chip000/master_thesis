% resumo
O processo evolutivo foi o principal responsável pela diferenciação
entre os seres vivos e uma das teorias contemporâneas acerca do modo
como ocorre esse processo afirma que, durante o curso da evolução,
mudanças genéticas aconteceram, criando diferentes espécies de seres
vivos. Muitas dessas mudanças são devido a mutações pontuais que
alteram a cadeia de DNA, impedindo que a informação seja expressa, ou
que seja expressa de um modo diferente. A comparação de sequência é o
método mais usual de se caracterizar a ocorrência de mutações
pontuais, sendo um dos problemas mais abordados em Biologia
Computacional. O interesse em fazer tal comparação é encontrar a
distância de edição~\cite{SetubalMeidanis*1997}. A distância de edição
é uma medida capaz de estimar a distância evolutiva entre duas
cadeias, mas não possui a informação de quais operações globais foram
utilizadas para a transformação de uma sequência em outra. Estas
operações globais são os chamados Rearranjos de Genomas, que podem
ser, por exemplo, reversões, transposições, fissões e fusões. Um
conceito de distância pode ser definido para qualquer classe de
rearranjo como sendo o menor número de operações pertencentes a essa
classe que são necessários para transformar um genoma em outro. Por
exemplo, chama-se a distância de reversão o menor número de reversões
necessárias para transformar um genoma em
outro~\cite{BafnaPevzner*1996} e a distância de transposição é o menor
número de transposições~\cite{BafnaPevzner*1998}. Este trabalho segue
a linha de pesquisa utilizada por Dias e Dias~\cite{DiasDias*2009} e
nós apresentaremos modelos de Programação por Restrições para
ordenação por reversões e ordenação por reversões e transposições,
baseados na teoria do Problema de Satisfação de Restrições e na teoria
do Problema de Otimização com Restrições. Nós fizemos comparações com
os modelos de Programação por Restrições para ordenação por
transposições, descrito por Dias e Dias~\cite{DiasDias*2009},
ordenação por reversões e ordenação por reversões e transposições com
os modelos de Programação Linear Inteira para os mesmos problemas
descritos em Dias e Souza~\cite{DiasSouza*2007}.
