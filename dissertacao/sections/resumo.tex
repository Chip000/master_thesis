% resumo
A teoria da seleção natural de Darwin afirma que os seres vivos atuais
descendem de ancestrais, e ao longo da evolução, mutações genéticas
propiciaram o aparecimento de diferentes espécies de seres vivos. Muitas
mutações são pontuais, alterando a cadeia de DNA, o que pode impedir que
a informação seja expressa, ou pode expressá-la de um modo diferente. A
comparação de sequências é o método mais usual de se identificar a
ocorrência de mutações pontuais, sendo um dos problemas mais abordados
em Biologia Computacional. Rearranjo de Genomas tem como objetivo
encontrar o menor número de operações que transformam um genoma em
outro. Essas operações podem ser, por exemplo, reversões, transposições,
fissões e fusões. O conceito de distância pode ser definido para estes
eventos, por exemplo, a distância de reversão é o número mínimo de
reversões que transformam um genoma em outro~\cite{BafnaPevzner*1996} e
a distância de transposição é o número mínimo de transposições que
transformam um genoma em outro~\cite{BafnaPevzner*1998}. Nós trataremos
os casos em que os eventos de reversão e transposição ocorrem de forma
isolada e os casos quando os dois eventos ocorrem simultaneamente, com o
objetivo de encontrar o valor exato para a distância. Nós criamos
modelos de Programação por Restrições para ordenação por reversões e
ordenação por reversões e transposições, seguindo a linha de pesquisa
utilizada por Dias e Dias~\cite{DiasDias*2009}. Nós apresentaremos os
modelos de Programação por Restrições para ordenação por reversões,
ordenação por transposições e ordenação por reversões e transposições,
baseados na teoria do Problema de Satisfação de Restrições e na teoria
do Problema de Otimização com Restrições. Nós fizemos comparações com os
modelos de Programação por Restrições para ordenação por transposições,
descrito por Dias e Dias~\cite{DiasDias*2009}, e com as formulações de
Programação Linear Inteira para ordenação por reversões, ordenação por
transposições e ordenação por reversões e transposições, descritos por
Dias e Souza~\cite{DiasSouza*2007}.

