\documentclass[12pt,twoside]{report} % Use esta linha para tese com mais de 100 p{\'a}ginas
%    \documentclass[12pt]{report} % Use esta linha para tese com at{\'e} 100 p{\'a}ginas

% Definições do IC-Unicamp
\usepackage{ic-tese}

% Definições latex para portugues
\usepackage[utf8]{inputenc}
\usepackage[brazil]{babel}
\usepackage[T1]{fontenc}

% Formatação de URL
\usepackage{url}

% Adição de figuras no documento
\usepackage{graphicx}

% Tabelas longas (mais de uma página)
\usepackage{longtable}

% Formatação de códigos
\usepackage[algoruled]{algorithm2e}

% Definições matemáticas
\usepackage{amsmath}
\usepackage{amsfonts}

% Hiperlinks
\usepackage[pdftex]{hyperref} 

% Definições de hifenização
\hyphenation{a-de-qüa-dos a-li-nha-dos a-li-nha-men-to
  a-li-nha-men-tos a-lign-ments a-tra-vés BAliBASE e-xem-plo
  ge-ne-ra-li-za-da glo-bal-men-te pu-bli-ca-das re-fe-rên-cia
  re-pe-ti-çõees se-me-lhan-ça UPGMA vi-zi-nhas}

\newcommand{\aspas}[1]{``#1''}
\newcommand{\estr}[1]{\textsl{\foreignlanguage{american}{#1}}}

\newcommand{\ic}{Instituto de Computação}
\newcommand{\bkp}{\textit{breakpoints}}
\newcommand{\pli}{programação linear inteira}
\newcommand{\PLI}{Programação Linear Inteira}
\newcommand{\pr}{programação lógica por restrições}
\newcommand{\PR}{Programação Lógica por Restrições}
\newcommand{\rg}{rearranjo de genomas}
\newcommand{\unicamp}{Universidade Estadual de Campinas}
\newcommand{\seq}{seqüencia}

\newtheorem{teo}{Teorema}[chapter]

\begin{document}
  
\title{Programação Linear Inteira e Programação Lógica por Restrições
  para Problemas de Rearranjo de Genomas}

\author{Victor de Abreu Iizuka}

\degreesought{Mestrado} % Doutorado/Mestrado
\titlesought{Mestre}     % Doutor(a)/Mestre

\principaladvisor{Zanoni Dias}
\advisortitle{Orientador} % Orientadora

%\coadvisor{Byron Meter (Co-orientador)} % Pode ser omitido.

\firstreader{?}
\secondreader{?}
\thirdreader{?}
\fourthreader{?}

\grants{{\rm Suporte financeiro de: Bolsa da CNPq (processo (VERIFICAR) XYZ) 2007--2011 } } % Pode ser omitido.

%\submitdate{Abril de 1997} % Pode ser omitido.
%\date{12 de junho de 1997} % Pode ser omitido.

%   Se pretende utilizar \copyrighttrue, consulte antes seu orientador ou a CPG!!!
%   Este ponto envolve quest{\~o}es legais complexas e de graves conseq{\"u}{\^e}ncias.
\copyrightfalse % Caso nao queira que apareca a pagina de Copyright.
%    \copyrighttrue % Caso queira que apareca a pagina de Copyright.
%
%    \finalversiontrue % Caso seja a versao final.
\finalversionfalse % Caso nao seja a versao final ainda.
%
\tablespagetrue % Caso queira que apareca a Lista de Tabelas.
%    \tablespagefalse % Caso nao queira que apareca a Lista de Tabelas.
%
\figurespagetrue % Caso queira que apareca o Lista de Figuras.
%    \figurespagefalse % Caso nao queira que apareca o Lista de Figuras.
%

\beforepreface

\prefacesection{Agradecimentos}
Gostaria de agradecer aos meus pais que fizeram tudo de possível para
eu conseguir chegar até aqui, dando apoio e aquele ``puxão de orelha''
sempre quando precisou, ao meu irmão que sempre me ajuda quando
preciso e a todos os familiares que me apoiaram durante esse período.

Agradeço também ao professor Zanoni por ter auxiliado em todo o
momento em que estive com dúvidas e pela paciência dispensada durante
a orientação.

Finalmente, agradeço a todos os meus amigos, porque sem eles
certamente não seria possível continuar esse trabalho.


\prefacesection{Resumo}
TODO: O resumo deve conter no m{\'a}ximo 500 palavras...
% \include{sections/resum}

\prefacesection{Abstract}
TODO: The abstract must contain at most 500 words...



\afterpreface % Gera: Conteudo, Lista de Tabelas, Lista de Figuras.
%

\chapter{Introdução}
\label{cap:intro}
% introducao
O processo evolutivo foi o principal responsável pela diferenciação
entre os seres vivos e uma das teorias contemporâneas acerca do modo
como ocorre esse processo afirma que, durante o curso da evolução,
mudanças genéticas aconteceram criando diferentes espécies de
seres vivos.

Muitas dessas mudanças são devido a mutações pontuais que alteram a
cadeia de DNA, impedindo que a informação seja expressa, ou que seja
expressa de um modo diferente. Tais alterações debilitam, na maioria
dos casos, o organismo portador ou proporcionam vantagens no processo
de seleção natural.

A comparação de \seq{} é o método mais usual de se caracterizar a
ocorrência de mutações pontuais, sendo um dos problemas mais abordados
em Biologia Computacional. O interesse em fazer tal comparação é
encontrar a distância de edição \cite{SetubalMeidanis*1997}, que é o
número mínimo de operações de inserção, remoção e substituição
necessárias que transformam uma sequência em outra.

A distância de edição é uma medida capaz de estimar a distância
evolutiva entre duas cadeias, mas não possui a informação de quais
operações globais foram utilizadas para a transformação de uma
\seq{} em outra. Estas operações globais são os chamados
Rearranjos de Genomas, que podem ser, por exemplo, reversões,
transposições, fissões e fusões.

Um conceito de distância pode ser definido para qualquer classe de
rearranjo como sendo o menor número de operações pertencentes a essa
classe que são necessários para transformar um genoma em outro. Por
exemplo, chama-se a distância de reversão o menor número de reversões
necessárias para transformar um genoma em outro
\cite{BafnaPevzner*1996} e a distância de transposição é o menor
número de transposições \cite{BafnaPevzner*1998}.

Estudos mostram que os rearranjos de genomas são mais apropriados que
mutações pontuais quando se deseja comparar os genoma de duas espécies
\cite{PalmerHerbon*1988}. Nesse contexto, a distância evolutiva entre
dois genomas pode ser estimada pelo conceito de distância para uma
classe de rearranjo definido no parágrafo anterior.

Neste trabalho trataremos os casos em que os eventos de transposição e
reversão ocorrem de forma isolada e os casos quando os dois eventos
ocorrem ao mesmo tempo.

% transposição
O evento de transposição ocorre quando dois blocos adjacentes no
genoma trocam de posição. O problema da distância de transposição é
encontrar o número mínimo de transposições necessárias que transforma
um genoma em outro. A complexidade deste problema ainda está em
aberto, não conhecemos nenhuma prova que mostre que ele se encontra na
classe dos problemas NP-Difícil e não existem evidências da existência
de um algoritmo polinomial. O melhor algoritmo de aproximação
conhecido possui razão de $1.375$ apresentado por Elias e
Hartman \cite{EliasHartman*2006}.

% reversão
O evento de reversão ocorre quando um bloco do genoma é invertido. O
problema da distância de reversão é encontrar o número mínimo de
reversões necessárias que transforma um genoma em outro. Neste
problema é importante saber se a orientação dos genes é conhecida,
pois existem algoritmos polinomiais para este caso. Entretanto, se não
se conhece a orientação dos genes o problema da distância de reversão
pertence a classe de problemas NP-Difícil, com a prova apresentada por
Caprara \cite{Caprara*1997}, e o melhor algoritmo de aproximação
conhecido possui razão de $1.375$ apresentado por Berman,
Hannenhalli e Karpinski \cite{BermanHannenhalliKarpinski*2002}.

% reversão+transposição
Na natureza um genoma não sofre apenas eventos de reversão ou de
transposição, ele está exposto a diversos eventos diferentes. Para
esta situação, iremos estudar o caso onde os eventos de reversão e
transposição ocorrem simultaneamente sobre um genoma. Os trabalhos de
Walter, Dias e
Meidanis \cite{MeidanisWalterDias*2002,WalterDiasMeidanis*1998} e Lin
e Xue \cite{LinXue*1999} estudaram o problema de encontrar o número
mínimo de reversões e transposições necessárias para transformar um
genoma em outro.

O trabalho desenvolvido nesta dissertação segue a linha de pesquisa
utilizada por Dias e Dias \cite{DiasDias*2009} e nós apresentaremos
modelos de Programação por Restrições (CP\footnote{Do
inglês \estr{Constraint Programming}}.) para ordenação por reversões e
ordenação por reversões e transposições, baseados na teoria do
Problema de Satisfação de Restrições (CSP\footnote{Do
inglês \estr{Constraint Satisfaction Problems}}.) e na teoria do
Problema de Otimização com Restrições (COP\footnote{Do
inglês \estr{Constraint Optimization Problems}}.). Nós fizemos
comparações com os modelos de Programação por Restrições para
ordenação por transposições, descrito por Dias e
Dias \cite{DiasDias*2009}, ordenação por reversões e ordenação por
reversões e transposições com os modelos de Programação Linear Inteira
(PLI\footnote{Do inglês \estr{Integer Linear Programming}}.) para os
mesmos problemas descritos em Dias e Souza \cite{DiasSouza*2007}.

O texto da dissertação está dividido da seguinte maneira. O
Capítulo \ref{sec:basic} apresenta uma série de conceitos e definições
sobre os eventos que foram usados neste trabalho. O
Capítulo \ref{sec:model} descreve detalhadamente os modelos usados
neste trabalho. O Capítulo \ref{sec:resul} traz a análise dos
resultados obtidos durante o trabalho. O Capítulo \ref{sec:concl}
apresenta as conclusões da dissertação.


\chapter{Conceitos Básicos}
\label{cap:basic}
% conceitos basicos
Neste capítulo faremos uma apresentação dos conceitos básicos
necessários para o entendimento e desenvolvimento deste trabalho. Na
seção \ref{sec:defin} mostraremos as definições usadas no decorrer
deste trabalho. As seções \ref{sec:rev}, \ref{sec:trans}
e \ref{sec:rev_trans} descrevem, respectivamente, os problemas de
ordenação por reversões, ordenação por transposições e ordenação por
reversões e transposições.

\section{Definições}
\label{sec:defin}
Para todos os problemas usamos as seguintes definições.

\textit{Permutação.} 
Para fins computacionais, um genoma é representado por uma $n$-tupla
de genes, e quando não há genes repetidos essa $n$-tupla é chamada de
permutação. Uma permutação é representada como $\pi =
(~\pi_{1}~\pi_{2}~\ldots~\pi_{n}~)$, para $\pi_{i} \in \mathbb{N}$, $0
< \pi_{i} \leq n$ e $i \neq j \leftrightarrow \pi_{i} \neq \pi_{j}$. A
permutação identidade é representada como $\iota =
(~1~2~3~\ldots~n~)$. Para a demonstração dos eventos, usaremos como
base a permutação $\pi = (~4~7~3~6~2~5~1~)$.

\textit{Eventos de rearranjo.}
Os eventos de rearranjo tratados neste trabalho são os eventos de
transposição e reversão quando ocorrem isoladamente e quando ocorrem
de forma conjunta. Os eventos são representados por $\rho$ e são
aplicados a $\pi$ de uma maneira específica.

\section{Ordenação por Reversões}
\label{sec:rev}
Um evento de reversão ocorre quando um bloco do genoma é
invertido. Uma reversão $\rho(i, j)$, para $1 \leq i < j \leq n$,
aplicada ao genoma $\pi = (~\pi_{1}~\pi_{2}~\ldots~\pi_{n}~)$ gera a
permutação $\rho\pi =
(\pi_{1}~\ldots~\pi_{i-1}~\pi_{j}~\pi_{j-1}~\ldots~\pi_{i+1}$
$\pi_{i}~ \pi_{j+1}~\ldots~\pi_{n})$, caso a orientação de $\pi$ não é
conhecida (Figura \ref{fig:rev_nao_orientada}), e $\rho\pi =
(\pi_{1}~\ldots~\pi_{i-1}~-\pi_{j}~-\pi_{j-1}~\ldots~-\pi_{i+1}$
$-\pi_{i}~ \pi_{j+1}~\ldots~\pi_{n})$, caso a orientação de $\pi$ é
conhecida (Figura \ref{fig:rev_orientada}). 

\begin{figure}
  \centering
  \includegraphics{images/rev_nao_orientada.png} 
  \caption{Reversão em uma permutação não orientada.}
  \label{fig:rev_nao_orientada}
\end{figure}

\begin{figure}
  \centering
  \includegraphics{images/rev_orientada.png}
  \caption{Reversão em uma permutação orientada.}
  \label{fig:rev_orientada}
\end{figure}

A distância de reversão $d_{r}(\pi,\sigma)$ entre duas permutações
$\pi$ e $\sigma$ é o número mínimo $r$ de reversões $\rho_{1}$,
$\rho_{2}$, $\ldots$, $\rho_{r}$ tal que
$\pi \rho_{1} \rho_{2} \ldots \rho_{r} = \sigma$. Note que a distância
de reversão entre $\pi$ e $\sigma$ é igual à distância de reversão
entre $\sigma^{-1} \pi$ e $\iota$. Então, sem perda de generalidade,
podemos dizer que o problema da distância de reversão é equivalente ao
problema de ordenação por reversões, que é a distância de reversão
entre a permutação $\pi$ e a permutação identidade $\iota$, denotado
por $d_{r}(\pi)$.

Em um estudo inicial sobre este problema, Bafna e
Pevzner \cite{BafnaPevzner*1996} apresentaram um algoritmo de
aproximação com razão de 1.5 quando a orientação de genes é conhecida
e 1.75 caso contrário.

Conhecer a orientação dos genes em um genoma é um fator importante no
problema de reversão, pois existem algoritmos polinomiais para o caso
em que a orientação é conhecida. No caso em que não se conhece a
orientação dos genes o problema de encontrar a distância de reversão
pertence a classe dos problemas NP-Difíceis \cite{Caprara*1997}.

O primeiro algoritmo polinomial para o problema de reversão com
orientação conhecida foi criado por Hannenhalli e Pevzner
\cite{HannenhalliPevzner*1995} que fez uso de várias operações
aplicadas a uma estrutura intermediária conhecida como grafo
de \bkp{}. A estratégia usada por Hannenhalli e Pevzner foi
simplificada no trabalho de Bergeron \cite{Bergeron*2005}. Atualmente
já existe um algoritmo com complexidade sub-quadrática
\cite{TannierSagot*2004} e, quando apenas a distância é necessária, um
algoritmo linear pode ser usado \cite{BaderMoretYan*2001}.

Um resultado importante obtido por Meidanis, Walter e Dias
\cite{MeidanisWalterDias*2000}, mostrou que toda teoria sobre
reversões desenvolvida para genomas lineares pode ser adaptada
facilmente para genomas circulares, que são comuns em seres inferiores
como vírus e bactérias.

Quando a orientação dos genes não é conhecida existem algoritmos de
aproximação que seguiram a ideia do trabalho de Bafna e Pevzner citado
anteriormente como, por exemplo, o algoritmo implementado por Berman,
Hannenhalli e Karpinski \cite{BermanHannenhalliKarpinski*2002} com
razão de aproximação de 1.375.

\subsection{Grafo de \bkp{}}
\label{subsec:rev_bkp}
O conceito de grafo de \bkp{} foi introduzido no trabalho de Bafna e
Pevzner \cite{BafnaPevzner*1996}. Inicialmente a permutação $\pi$ é
estendida adicionando o elemento $\pi_{0} = 0$ e $\pi_{n+1} =
n+1$. Dois elementos consecutivos $\pi_{i}$ e $\pi_{i+1}$, $0 \le
i \le n$, são \textit{adjacentes} quando $|\pi_{i} - \pi_{i+1}| = 1$,
e são \estr{breakpoints} caso contrário. Define-se um grafo de arestas
coloridas $G(\pi)$ com $n + 2$ vértices \{0, 1, $\ldots$, $n$, $n +
1$\}. Unimos os vértices $i$ e $j$ com uma aresta preta se $(i, j)$
for um \estr{breakpoint}. Unimos os vértices $i$ e $j$ com uma aresta
cinza se $|i - j| = 1$ e $i$, $j$ não são consecutivos em
$\pi$. Denotamos por $b_r(\pi)$ o número de \bkp{} existentes em
$\pi$. A Figura \ref{fig:rev_grafo_bkp} mostra o grafo de \bkp{} para
a permutação $\pi = (~4~7~3~6~2~5~1~)$.

\begin{figure}[h]
  \centering 
  \includegraphics[scale=0.5]{images/rev_grafo_bkp.png} 
  \caption{Grafo de \bkp{} para a permutação $\pi = (~4~7~3~6~2~5~1~)$.}
  \label{fig:rev_grafo_bkp}
\end{figure}

\subsection{Limitantes}
\label{subsec:rev_limitantes}
Para o problema de ordenação por reversões usamos dois tipos de limitantes:

\begin{itemize}
\item{\textit{rev\_def}. 
Este é o limitante padrão, com limite inferior igual à 0 e limite
superior igual a $n$ (tamanho da permutação).}
\item{\textit{rev\_br}
Usando o conceito de \bkp{}, temos que uma reversão atua em dois
pontos em uma permutação e, portanto, pode reduzir o número de \bkp{}
em pelo menos um e no máximo dois \cite{BafnaPevzner*1996}, levando ao
Teorema \ref{teo:rev_bound}.

\begin{teo}
\label{teo:rev_bound}
Para qualquer permutação $\pi$, $\frac{b_r(\pi)}{2} \leq d_r(\pi) \leq
  b_r(\pi)$.
\end{teo}}
\end{itemize}

\section{Ordenação por Transposições}
\label{sec:trans}

Um evento de transposição ocorre quando dois blocos adjacentes no
genoma trocam de posição.

TODO

\section{Ordenação por Reversões e Transposições}
\label{sec:rev_trans}
TODO


\chapter{Modelos}
\label{cap:model}
TODO: Descrição dos Modelos usados para CP e ILP
% Neste capítulo nós apresentaremos a descrição dos modelos de \pr{} e
\pli{} usados para os problemas de ordenação por transposições,
ordenação por reversões e ordenação por reversões e transposições.

\section{\PR}
\label{sec:cp}
O modelo de \pr{} usado para o problema de ordenação por transposições
é o descrito em Dias e Dias \cite{DiasDias*2009}. Nós usamos os
limitantes inferior e superior descritos em (REFERENCIAR) para
escrever as formulações baseadas nas teorias de Problema de Satisfação
de Restrições (CSP) e Problema de Otimização com Restrições (COP). As
formulações foram descritas usando a notação prolog-like de
Marriot \cite{Marriott*1998}. Primeiramente iremos apresentar os
predicados que são comum às duas formulações.

Em prolog as variáveis são descritas por \textit{strings} iniciadas
com letra maiúscula ou ``\_'' (\textit{underscore}) caso a variável
seja anônima. As letras gregas $\pi$ e $\sigma$ representam listas
nesta notação. A construção $X~::~[i~..~j]$ significa que $X$ (ou cada
elemento de $X$ se $X$ for uma lista) pode assumir um valor do
intervalo $[i~..~j]$.

A representação da permutação (\ref{perm}) e o efeito das operações de
reversão (\ref{reversal}) e transposição (\ref{transposition} podem
ser vistas da mesma maneira que são descritas pelos problemas. Neste
modelo a permutação $\pi$ é uma lista de elementos
($\pi_{1},~\pi_{2},~\ldots~,~\pi_{n}$) onde $\pi_{i} \in \mathbb{N}$,
$0 < \pi_{i} \le n$ e $\pi_{i} \neq \pi_{j}$ para $i \neq j$.
\begin{align}
  \label{perm}
  \textit{per}&\textit{mutation}(\pi, N)~\text{:-} \nonumber\\
  &\textit{length}(\pi, N), \\ 
  &\pi~::~[1~..~N], \nonumber\\
  &\textit{all\_different}(\pi). \nonumber
\end{align}

Na reversão $\rho(i,j)$, $0 < i < j \leq n$, dividimos a lista em três
sublistas $C_{1}C_{2}C_{3}$ onde $C_{1} = (\pi_{1}~..~\pi_{i-1})$,
$C_{2} = (\pi_{i}~..~\pi_{j})$ e $C_{3} =
(\pi_{j+1}~..~\pi_{n})$. Depois fazemos a reversão na sublista
$C_{2}$, resultando na lista $R_{C_{2}}$. Então juntamos a nova lista
$R_{C_{2}}$ com as sublistas $C_{1}$ e $C_{3}$ para formar $\rho\pi =
C_{1}R_{C_{2}}C_{3}$.
\begin{align}
  \label{reversal}
  \textit{rev}&\textit{ersal}(\pi, \sigma, I, J)~\text{:-} \nonumber\\
  &\textit{permutation}(\pi, N), \nonumber\\
  &\textit{permutation}(\sigma, N), \nonumber \\
  &1 \le I < J \le N, \\
  &\textit{split}(\pi, I, J, C_{1}, C_{2}, C_{3}), \nonumber\\
  &\textit{reverse}(C_{2}, R_{C_{2}}), \nonumber \\
  &\sigma = C_{1}, R_{C_{2}}, C_{3}. \nonumber
\end{align}

Na transposição $\rho(i,j,k)$, $0 < i < j < k\leq n$, dividimos a
lista em quatro sublistas $C_{1}C_{2}C_{3}C_{4}$ onde $C_{1} =
(\pi_{1}~..~\pi_{i-1})$, $C_{2} = (\pi_{i}~..~\pi_{j-1})$, $C_{3} =
(\pi_{j}~..~\pi_{k-1})$ e $C_{4} = (\pi_{k}~..~\pi_{n})$. Trocamos de
posição os blocos $C_{2}$ e $C_{3}$ e juntamos elas na ordem $C_{1}$,
$C_{3}$, $C_{2}$ e $C_{4}$ para formar $\rho\pi =
C_{1}C_{3}C_{2}C_{4}$. Observe que as sublistas $C_{1}$ e $C_{4}$
podem ser vazias.
\begin{align}
  \label{transposition}
  \textit{tra}&\textit{nsposition}(\pi, \sigma, I, J, K)~\text{:-} \nonumber\\
  &\textit{permutation}(\pi, N), \nonumber\\
  &\textit{permutation}(\sigma, N), \\
  &1 \le I < J < K \le N, \nonumber \\
  &\textit{split}(\pi, I, J, K, C_{1}, C_{2}, C_{3}, C_{4}), \nonumber\\
  &\sigma = C_{1}, C_{3}, C_{2}, C_{4}. \nonumber
\end{align}

\subsection{Modelo CSP}
\label{subsec:modelcsp}
Primeiramente modelaremos o problema usando a teoria CSP, mas o número
de variáveis é desconhecido devido ao fato de precisarmos do valor da
distância $d_{r}(\pi)$ para criar as restrições e variáveis que
representam as permutações. Por esta razão, nós escolhemos um valor
candidato para a distância $R$ tal que $R \in [LB~..~UB]$, onde $LB$ é
um limitante inferior conhecido e $UB$ é um limitante superior
conhecido para o problema, e tentamos achar a combinação apropriada de
$R$ reversões que solucionam o problema. Se o modelo CSP falha (não
existe combinação que soluciona o problema com o valor candidato
escolhido) com o candidato $R$, nós escolhemos outro valor $R$ apenas
incrementando seu valor. O valor de $R$ é escolhido usando uma
estratégia \textit{bottom-up}\footnote{EXPLICAR BOTTOM-UP} e por
definição não verificamos nenhum valor maior que o limitante superior
$UB$. Na transposição, o processo é o mesmo que na reversão, trocando
apenas o valor da distância de reversão ($d_{r}(\pi)$) para o valor da
distância de transposição ($d_{t}(\pi)$).
\begin{align}
  \label{revdistance}
  \textit{rev}&\textit{ersal\_distance}(\iota, 0, \_Model). \nonumber\\
  \textit{rev}&\textit{ersal\_distance}(\pi, R, Model)~\text{:-} \nonumber\\
  &\textit{bound}(\pi, Model, LB, UB), \nonumber\\
  &R :: [LB .. UB], \\
  &\textit{indomain}(R), \nonumber \\
  &\textit{reversal}(\pi, \sigma, \_I, \_J), \nonumber \\
  &\textit{reversal\_distance}(\sigma, R-1, Model). \nonumber
\end{align}
\begin{align}
  \label{tradistance}
  \textit{tra}&\textit{nsposition\_distance}(\iota, 0, \_Model). \nonumber\\
  \textit{tra}&\textit{nsposition\_distance}(\pi, T, Model)~\text{:-} \nonumber\\
  &\textit{bound}(\pi, Model, LB, UB), \nonumber\\
  &T :: [LB .. UB], \\
  &\textit{indomain}(T), \nonumber \\
  &\textit{transposition}(\pi, \sigma, \_I, \_J, \_K), \nonumber \\
  &\textit{transposition\_distance}(\sigma, T-1, Model). \nonumber
\end{align}

O predicado \textit{rev\_trans\_dist/3} (\ref{trarevdist}) retorna o
valor da distância de reversão e transposição. O
predicado \textit{event/2} escolhe o melhor evento entre o
predicado \textit{reversal/4} (\ref{reversal}) e o
predicado \textit{transposition/5} (\ref{transposition}) para
minimizar o valor da distância.
\begin{align}
  \label{trarevdist}
  \textit{rev}&\textit{\_trans\_dist}(\iota, 0, \_Model). \nonumber\\
  \textit{rev}&\textit{\_trans\_dist}(\pi, N, Model)~\text{:-} \nonumber\\
  &\textit{bound}(\pi, Model, LB, UB), \nonumber\\
  &N :: [LB .. UB], \\
  &\textit{indomain}(N), \nonumber \\
  &\textit{event}(\pi, \sigma), \nonumber \\
  &\textit{rev\_trans\_dist}(\sigma, N-1, Model). \nonumber
\end{align}

O predicado \textit{indomain(X)} em (\ref{revdistance}),
(\ref{tradistance}) e (\ref{trarevdist}) pega o domínio da variável
$X$ e escolhe o menor elemento dele (no caso o valor do limitante
inferior). Se o modelo retorna para o predicado \textit{indomain}
devido a uma falha, o elemento que originou ela será removido do
domínio e um outro valor será escolhido.

Os modelos CSP para os problemas possuem a estrutura mostrada acima,
trocando apenas os limitantes usados. Em comum a todas operações temos
o modelo \textit{def\_csp} que não usa nenhum limitante.

\textit{Ordenação por reversões:}
\begin{itemize}
\item{\textit{rev\_br\_csp}: 
Modelo que usa o conceito de \bkp{} em reversões para calcular os
limitantes conforme descrito ..... (REFERENCIA)}
\item{\textit{rev\_cg\_csp}:
Modelo que usa o conceito de decomposição de 2-ciclos no grafo de
ciclos em reversões para calcular os limitantes conforme descrito
.... (REFERENCIA)}
\end{itemize}

\textit{Ordenação por transposições:}
\begin{itemize}
\item{\textit{tra\_br\_csp}: 
Modelo que usa o conceito de \bkp{} em transposições para calcular os
limitantes conforme descrito .... (REFERENCIA)}
\item{\textit{tra\_cg\_csp}:
Modelo que usa o conceito de grafo de ciclos em transposições, fazendo
a decomposição de ciclos e analisando os ciclos ímpares separadamente
para calcular os limitantes conforme descrito .... (REFERENCIA)}
\end{itemize}

\textit{Ordenação por reversões e transposições:} 
\begin{itemize}
\item{\textit{r\_t\_br\_csp}: 
Melhor limitante superior entre o limitante de \bkp{} para reversões e
o limitante de \bkp{} para transposições.}
\item{\textit{r\_t\_bc\_csp}:
Melhor limitante superior entre o limitante de \bkp{} para reversões e
o limitante do grafo de ciclos para transposições.}
\item{\textit{r\_t\_cc\_csp}: 
Melhor limitante superior entre o limitante do grafo de ciclos para
reversões e o limitante do grafo de ciclos para transposições.}
\end{itemize}

O predicado \textit{bound/4} (\ref{bound}) recebe na
variável \textit{Model} um átomo que representa o modelo a ser
usado. Este átomo se conecta com o predicado que retorna o limitante
superior e inferior apropriado para o modelo. Observe que o limitante
inferior é igual a $0$ no caso dos modelos de ordenação por reversões
e transposições. Isto ocorre devido ao fato que a cada nova iteração
do modelo pode surgir um limitante inferior melhor, simplesmente
fazendo a troca entre as operações de reversão e transposição.
\begin{align}
  \label{bound}
  \textit{bou}&\textit{nd}(\pi, def\_csp, LB, UB)~\text{:-} \nonumber\\
  &\textit{def\_csp\_bound}(\pi, LB, UB). \nonumber \\
  \textit{bou}&\textit{nd}(\pi, rev\_br\_csp, LB, UB)~\text{:-} \nonumber \\
  &\textit{calc\_breakpoints\_reversal}(\pi, B), \nonumber\\
  &LB = B / 2 \nonumber \\ % / 2 ,  \\
  &UB = B. \nonumber \\
  \textit{bou}&\textit{nd}(\pi, rev\_cg\_csp, LB, UB)~\text{:-} \nonumber \\
  &\textit{find\_2\_cycle}(\pi, B, C), \nonumber\\
  &LB = (2 * B - C) / 3 , \nonumber  \\
  &UB = B - C / 2. \nonumber \\
  \textit{bou}&\textit{nd}(\pi, tra\_br\_csp, LB, UB)~\text{:-} \nonumber\\
  &\textit{calc\_breakpoints\_transposition}(\pi, B), \nonumber\\
  &LB = B / 3 \nonumber \\ %/ 3, \nonumber \\
  &UB = B.  \nonumber \\
  \textit{bou}&\textit{nd}(\pi, tra\_br\_csp, LB, UB)~\text{:-} \\
  &\textit{calc\_oddcycle\_transposition}(\pi, N, C), \nonumber\\
  &LB = (N + 1 - C) / 2 \nonumber \\ 
  &UB = (3 * (N + 1 - C)) / 4. \nonumber \\
  \textit{bou}&\textit{nd}(\pi, r\_t\_br\_csp, 0, UB)~\text{:-} \nonumber\\
  &\textit{bound}(\pi, rev\_br, \_RLB, RUB), \nonumber\\
  &\textit{bound}(\pi, tra\_br, \_TLB, TUB), \nonumber\\
  &\textit{min}(RUB, TUB, UB), \nonumber\\
  \textit{bou}&\textit{nd}(\pi, r\_t\_bc\_csp, 0, UB)~\text{:-} \nonumber\\
  &\textit{bound}(\pi, rev\_br, \_RLB, RUB), \nonumber\\
  &\textit{bound}(\pi, tra\_cg, \_TLB, TUB), \nonumber\\
  &\textit{min}(RUB, TUB, UB), \nonumber\\
  \textit{bou}&\textit{nd}(\pi, r\_t\_cc\_csp, 0, UB)~\text{:-} \nonumber\\
  &\textit{bound}(\pi, rev\_cg, \_RLB, RUB), \nonumber\\
  &\textit{bound}(\pi, tra\_cg, \_TLB, TUB), \nonumber\\
  &\textit{min}(RUB, TUB, UB), \nonumber
\end{align}

\subsection{Modelo COP}
\label{subsec:modelcop}
Uma outra alternativa é modelar o problema usando a teoria COP. Os
modelos que usam esta abordagem necessitam de um limitante superior,
portanto serão feitas algumas alterações nos predicados definidos
anteriormente. Nós usamos as variáveis binárias $B$ para indicar
quando uma operação de reversão ou de transposição modificou ou não a
permutação fornecida.

O primeiro predicado que precisamos criar é o \textit{reversal\_cop/5}
(\ref{reversal_cop}). Primeiramente, dado uma reversão $\rho(i, j)$,
adicionamos uma nova restrição para permitir $(i, j) = (0, 0)$. Se
$(i, j) = (0, 0)$ então $\pi\rho = \pi$. Então, adicionamos um novo
argumento ao predicado \textit{reversal\_cop} que recebe a variável
$B$.
\begin{align}
  \label{reversal_cop}
  \textit{rev}&\textit{ersal\_cop}(\iota, \iota, 0, 0, 0). \\
  \textit{rev}&\textit{ersal\_cop}(\pi, \sigma, I, J, 1)~\text{:-}~ 
  \textit{reversal}(\pi, \sigma, I, J). \nonumber
\end{align}

O predicado equivalente para transposição é
o \textit{transposition\_cop/6} (\ref{transposition_cop}). Neste caso,
dado uma transposição $\rho(i, j, k)$, adicionamos uma nova restrição
para permitir $(i, j, k) = (0, 0, 0)$. Se $(i, j, k) = (0, 0, 0)$
então $\pi\rho = \pi$.
\begin{align}
  \label{transposition_cop}
  \textit{tra}&\textit{nsposition\_cop}(\iota, \iota, 0, 0, 0, 0). \\
  \textit{tra}&\textit{nsposition\_cop}(\pi, \sigma, I, J, K, 1)~\text{:-}~ 
  \textit{transposition}(\pi, \sigma, I, J, K). \nonumber
\end{align}

Para calcular a distância nos modelos baseados na teoria COP,
implementamos o predicado \textit{reversal\_distance\_cop/3}
(\ref{revdistance_cop}), que ajusta as variáveis $B$ usando o valor do
limitante superior e restringe as permutações fazendo $\pi_{k}
= \pi_{k-1} \rho_{k}$. O predicado \textit{length/2}, predicado
interno do prolog, é usado para criar uma lista de variáveis não
instanciadas com o tamanho dado. A função de custo \textit{Cost} é a
soma das variáveis $B$ associadas com cada $\rho_{k}$, $Cost
= \sum_{k=1}^{UB} B_{k}$, onde $UB$ é um limitante superior
conhecido. A distância de reversão é o valor mínimo da função de custo
$d_{r} = \min Cost$. Para evitar processamento desnecessários, o valor
de $Cost$ precisa ser maior ou igual a qualquer limitante inferior. O
predicado equivalente para transposição é
o \textit{transposition\_distance\_cop/3} (\ref{tradistance_cop}).
\begin{align}
  \label{revdistance_cop}
  \textit{rev}&\textit{ersal\_distance\_cop}(\pi, R, Model)~\text{:-} \nonumber\\
  &\textit{bound}(\pi, Model, LB, UB), \nonumber\\
  &\textit{length}(B, UB), \nonumber \\
  &\textit{upperbound\_constraint\_rev}(\pi, B, Model, UB), \\
  &\textit{sum}(B, Cost), \nonumber \\
  &\textit{Cost} \ge \textit{LB}, \nonumber \\
  &\textit{minimize}(Cost, R). \nonumber
\end{align}
\begin{align}
  \label{tradistance_cop}
  \textit{tra}&\textit{nsposition\_distance\_cop}(\pi, T, Model)~\text{:-} \nonumber\\
  &\textit{bound}(\pi, Model, LB, UB), \nonumber\\
  &\textit{length}(B, UB), \nonumber \\
  &\textit{upperbound\_constraint\_trans}(\pi, B, Model, UB), \\
  &\textit{sum}(B, Cost), \nonumber \\
  &\textit{Cost} \ge \textit{LB}, \nonumber \\
  &\textit{minimize}(Cost, T). \nonumber
\end{align}

O predicado equivalente para o modelo de ordenação por reversões e
transposições é
o \textit{rev\_trans\_dist\_cop/3}~(\ref{trarevdistcop}). O
predicado \textit{upperbound\_constraint\_event/4} escolhe o melhor
evento entre a reversão, usando o
predicado \textit{upperbound\_constraint\_rev/4}
(\ref{ub_constaint_rev}), e a transposição, usando o
predicado \textit{upperbound\_constraint\_trans/4}
(\ref{ub_constaint_tra}), para minimizar o valor da distância.
\begin{align}
  \label{trarevdistcop}
  \textit{rev}&\textit{\_trans\_dist\_cop}(\pi, N, Model)~\text{:-} \nonumber\\
  &\textit{bound}(\pi, Model, LB, UB), \nonumber\\
  &\textit{length}(B, UB), \nonumber \\
  &\textit{upperbound\_constraint\_event}(\pi, B, Model, UB), \\
  &\textit{sum}(B, Cost), \nonumber \\
  &\textit{Cost} \ge \textit{LB}, \nonumber \\
  &\textit{minimize}(Cost, N). \nonumber
\end{align}

O predicado \textit{upperbound\_constraint\_rev/4}
(\ref{ub_constaint_rev}) aplica na permutação os efeitos de $\rho_{k}$
e retorna o valor apropriado de $B$ para cada reversão $\rho_{k}$. Uma
restrição importante é verificar se é possível ordenar a permutação
usando o número restante de reversões para evitar processamento
desnecessário. O predicado equivalente para a transposição é o
\textit{upperbound\_constraint\_trans/4} (\ref{ub_constaint_tra}).
\begin{align}
  \label{ub_constaint_rev}
  \textit{upp}&\textit{erbound\_constraint\_rev}(\iota, [~], \_Model, \_UB). \nonumber\\
  \textit{upp}&\textit{erbound\_constraint\_rev}(\pi, [B|Bt], Model, UB)~\text{:-} \nonumber\\
  &\textit{reversal\_cop}(\pi, \sigma, \_I, \_J, B), \\
  &\textit{bound}(\pi, Model, LB, \_UB), \nonumber\\
  &UB \ge LB, \nonumber \\
  &\textit{upperbound\_constraint\_rev}(\sigma, Bt, Model, UB - 1), \nonumber 
\end{align}
\begin{align}
  \label{ub_constaint_tra}
  \textit{upp}&\textit{erbound\_constraint\_trans}(\iota, [~], \_Model, \_UB). \nonumber\\
  \textit{upp}&\textit{erbound\_constraint\_trans}(\pi, [B|Bt], Model, UB)~\text{:-} \nonumber\\
  &\textit{transposition\_cop}(\pi, \sigma, \_I, \_J, \_K, B), \\
  &\textit{bound}(\pi, Model, LB, \_UB), \nonumber\\
  &UB \ge LB, \nonumber \\
  &\textit{upperbound\_constraint\_trans}(\sigma, Bt, Model, UB - 1), \nonumber 
\end{align}

Os modelos baseados na teoria COP possuem a estrutura acima, trocando
apenas os limitantes usados. Os limitantes são os mesmos usados para
os modelos CSP, modificados para os modelos COP. Então temos os
seguintes limitantes: \textit{def\_cop}, \textit{rev\_br\_cop},
\textit{rev\_cg\_cop}, \textit{tra\_br\_cop}, \textit{tra\_cg\_cop}, 
\textit{r\_t\_br\_cop}, \textit{r\_t\_bc\_cop}
e \textit{r\_t\_cc\_cop}.

\section{\PLI}
\label{sec:cp}


\chapter{Análise dos Resultados}
\label{cap:resul}
TODO: Análise dos Resultados obtidos
% % resultados
Neste capítulo apresentaremos os resultados obtidos pelos modelos
descritos no capítulo~\ref{cap:model}. A seção \ref{sec:tspec} mostra
as características do computador utilizado para executar os testes. A
seção \ref{sec:testes} descreve como os testes foram executados. A
seção \ref{sec:analise} apresenta a análise dos resultados obtidos
durante este trabalho.

\section{Especificações Técnicas}
\label{sec:tspec}
O computador utilizado para executar os testes possui as seguintes
características:

\begin{itemize}
  \item{Processador: Intel\textregistered{}~Core\texttrademark~2 Duo
  2.33GHz.}

  \item{Memória RAM: 3 GB.}
  
  \item{Sistema Operacional: Ubuntu Linux com kernel 2.6.31.}
\end{itemize}

Todos modelos de \pr{} foram implementados usando as seguintes
ferramentas:

\begin{itemize}
  \item{Sistema de programação de código
  aberto \textit{ECLiPSe-6.0}~\cite{eclipse*2009}. } 
  
 \item{Pacote proprietário para a linguagem de programação
  C++ \textit{IBM\textregistered{} ILOG\textregistered{}
  CPLEX\textregistered{} CP Optimizer v 2.3}~\cite{ilogcp*2011}.}
\end{itemize}

Todas formulações de programação linear inteira foram implementadas
usando as seguintes ferramentas:

\begin{itemize}
  \item{Sistema de programação de código
  aberto \textit{GLPK-4.35}~\cite{glpk*2010}.}
  
  \item{Pacote proprietário para a linguagem de programação
  C++ \textit{IBM\textregistered{} ILOG\textregistered{}
  CPLEX\textregistered{} Optimizer v 12.1}~\cite{ilogcplex*2011}..}
\end{itemize}

\section{Descrição dos Testes}
\label{sec:testes}
Os testes foram separados de acordo com o tamanho das
permutações. Uma instância contém o conjunto de permutações com
tamanho $n$, onde $n > 2$ devido ao fato de ser trivial fazer uma das
operações de ordenação em uma permutação com tamanho $2$. Para cada
instância, geramos $50$ permutações com tamanho $n$.

Todas instâncias foram executadas nos softwares indicados na
seção \ref{sec:tspec}. Para cada instância foi dado o tempo máximo de
$25$ horas. Fazemos a comparação dos modelos se baseando nos tempos
médios usados para resolver cada instância. Como referência usamos os
modelos de \pli{} descritos na seção \ref{sec:pli}.

\section{Análise dos Resultados}
\label{sec:analise}





% DO NOT FORGET THE PACKAGE 'rotating'!!!!
\begin{sidewaystable}[!ht]
  \caption{Tempo médio (em segundos) para o modelo de ordernação por reversões. O caractere ``-'' significa que o modelo não conseguiu terminar o conjunto de testes dentro do limite de 25 horas.}
  \label{table:rev}
  \begin{center}
    \begin{tabular}{| r | r | r | r | r | r | r | r | r | r | r |}
      \hline
      \multicolumn{11}{|c|}{\textbf{Reversals Models}} \\
      \hline
      \textbf{size} & \multicolumn{8}{|c|}{\textbf{CP}} & \multicolumn{2}{|c|}{\textbf{ILP}} \\
      \cline{02-11}
        & \multicolumn{4}{|c|}{\textbf{ECLiPSe}} & \multicolumn{4}{|c|}{\textbf{ILOG CP}} & \textbf{GLPK} & \textbf{ILOG CPLEX}  \\
      \cline{02-09}
        & \textbf{~def\_cop~} & \textbf{~rev\_br\_cop~} & \textbf{~def\_csp~} & \textbf{~rev\_br\_csp~} & \textbf{~def\_cop~} & \textbf{~rev\_br\_cop~} & \textbf{~def\_csp~} & \textbf{~rev\_br\_csp~}  & & \\
      \hline
      ~3~ & ~0.009~ & ~0.004~ & ~0.003~ & ~0.004~ & ~0.002~ & ~0.001~ & ~0.004~ & ~0.003~ & ~0.001~ & ~0.002~ \\
      ~4~ & ~0.417~ & ~0.409~ & ~0.027~ & ~0.005~ & ~0.006~ & ~0.006~ & ~0.008~ & ~0.004~ & ~0.001~ & ~0.008~ \\
      ~5~ & ~35.502~ & ~43.286~ & ~0.280~ & ~0.010~ & ~0.018~ & ~0.017~ & ~0.016~ & ~0.011~ & ~0.096~ & ~0.036~ \\
      ~6~ & ~-~ & ~-~ & ~5.223~ & ~0.026~ & ~0.095~ & ~0.096~ & ~0.063~ & ~0.037~ & ~1.264~ & ~0.562~ \\
      ~7~ & ~-~ & ~-~ & ~490.753~ & ~0.226~ & ~1.494~ & ~1.356~ & ~0.334~ & ~0.261~ & ~4.702~ & ~16.011~ \\
      ~8~ & ~-~ & ~-~ & ~-~ & ~1.096~ & ~20.217~ & ~29.556~ & ~4.360~ & ~4.164~ & ~4.428~ & ~426.984~ \\
      ~9~ & ~-~ & ~-~ & ~-~ & ~6.885~ & ~989.500~ & ~1458.167~ & ~217.353~ & ~216.878~ & ~-~ & ~-~ \\
      ~10~ & ~-~ & ~-~ & ~-~ & ~30.742~ & ~-~ & ~-~ & ~-~ & ~-~ & ~-~ & ~-~ \\
      \hline
    \end{tabular}
  \end{center}
\end{sidewaystable}

% DO NOT FORGET THE PACKAGE 'rotating'!!!!
\begin{sidewaystable}[!ht]
  \caption{Tempo médio (em segundos) para o modelo de ordernação por transposições. O caractere ``-'' significa que o modelo não conseguiu terminar o conjunto de testes dentro do limite de 25 horas.}
  \label{table:trans}
  \begin{center}
    \scalebox{0.7}{
    \begin{tabular}{| r | r | r | r | r | r | r | r | r | r | r | r | r | r | r |}
      \hline
      \multicolumn{15}{|c|}{\textbf{Transpositions Models}} \\
      \hline
      \textbf{size} & \multicolumn{12}{|c|}{\textbf{CP}} & \multicolumn{2}{|c|}{\textbf{ILP}} \\
      \cline{02-15}
        & \multicolumn{6}{|c|}{\textbf{ECLiPSe}} & \multicolumn{6}{|c|}{\textbf{ILOG CP}} & \textbf{GLPK} & \textbf{ILOG CPLEX}  \\
      \cline{02-13}
        & \textbf{~def\_cop~} & \textbf{~tra\_br\_cop~} & \textbf{~tra\_cg\_cop~} & \textbf{~def\_csp~} & \textbf{~tra\_br\_csp~} & \textbf{~tra\_cg\_csp~} & \textbf{~def\_cop~} & \textbf{~tra\_br\_cop~} & \textbf{~tra\_cg\_cop~} & \textbf{~def\_csp~} & \textbf{~tra\_br\_csp~} & \textbf{~tra\_cg\_csp~}  & & \\
      \hline
      ~3~ & ~0.005~ & ~0.008~ & ~0.003~ & ~0.003~ & ~0.003~ & ~0.001~ & ~0.003~ & ~0.003~ & ~0.001~ & ~0.004~ & ~0.001~ & ~0.002~ & ~0.001~ & ~0.001~ \\
      ~4~ & ~0.553~ & ~1.704~ & ~0.066~ & ~0.021~ & ~0.007~ & ~0.001~ & ~0.006~ & ~0.005~ & ~0.005~ & ~0.008~ & ~0.006~ & ~0.004~ & ~0.001~ & ~0.011~ \\
      ~5~ & ~198.019~ & ~1479.668~ & ~3.276~ & ~0.335~ & ~0.023~ & ~0.004~ & ~0.041~ & ~0.014~ & ~0.010~ & ~0.029~ & ~0.021~ & ~0.011~ & ~0.196~ & ~0.057~ \\
      ~6~ & ~-~ & ~-~ & ~38.020~ & ~14.197~ & ~0.156~ & ~0.008~ & ~0.243~ & ~0.074~ & ~0.045~ & ~0.104~ & ~0.075~ & ~0.043~ & ~2.348~ & ~1.375~ \\
      ~7~ & ~-~ & ~-~ & ~-~ & ~920.502~ & ~1.416~ & ~0.029~ & ~1.588~ & ~1.253~ & ~0.429~ & ~0.559~ & ~0.472~ & ~0.266~ & ~4.650~ & ~79.015~ \\
      ~8~ & ~-~ & ~-~ & ~-~ & ~-~ & ~12.544~ & ~0.076~ & ~23.778~ & ~26.807~ & ~16.619~ & ~4.803~ & ~4.537~ & ~3.798~ & ~-~ & ~-~ \\
      ~9~ & ~-~ & ~-~ & ~-~ & ~-~ & ~49.813~ & ~0.382~ & ~1109.692~ & ~994.209~ & ~385.954~ & ~112.203~ & ~111.656~ & ~66.439~ & ~-~ & ~-~ \\
      ~10~ & ~-~ & ~-~ & ~-~ & ~-~ & ~1287.331~ & ~2.297~ & ~-~ & ~-~ & ~-~ & ~-~ & ~-~ & ~-~ & ~-~ & ~-~ \\
      \hline
    \end{tabular}
}
  \end{center}
\end{sidewaystable}

% DO NOT FORGET THE PACKAGE 'rotating'!!!!
\begin{sidewaystable}[!ht]
  \caption{Tempo médio (em segundos) para o modelo de ordernação por reversões e transposições. O caractere ``-'' significa que o modelo não conseguiu terminar o conjunto de testes dentro do limite de 25 horas.}
  \label{table:r_t}
  \begin{center}
    \scalebox{0.7}{
    \begin{tabular}{| r | r | r | r | r | r | r | r | r | r | r | r | r | r | r |}
      \hline
      \multicolumn{15}{|c|}{\textbf{Reversals and Transpositions Models}} \\
      \hline
      \textbf{size} & \multicolumn{12}{|c|}{\textbf{CP}} & \multicolumn{2}{|c|}{\textbf{ILP}} \\
      \cline{02-15}
        & \multicolumn{6}{|c|}{\textbf{ECLiPSe}} & \multicolumn{6}{|c|}{\textbf{ILOG CP}} & \textbf{GLPK} & \textbf{ILOG CPLEX}  \\
      \cline{02-13}
        & \textbf{~def\_cop~} & \textbf{~r\_t\_br\_cop~} & \textbf{~r\_t\_bc\_cop~} & \textbf{~def\_csp~} & \textbf{~r\_t\_br\_csp~} & \textbf{~r\_t\_bc\_csp~} & \textbf{~def\_cop~} & \textbf{~r\_t\_br\_cop~} & \textbf{~r\_t\_bc\_cop~} & \textbf{~def\_csp~} & \textbf{~r\_t\_br\_csp~} & \textbf{~r\_t\_bc\_csp~}  & & \\
      \hline
      ~3~ & ~0.034~ & ~0.012~ & ~0.003~ & ~0.004~ & ~0.003~ & ~0.002~ & ~0.004~ & ~0.002~ & ~0.001~ & ~0.004~ & ~0.002~ & ~0.003~ & ~0.001~ & ~0.002~ \\
      ~4~ & ~7.370~ & ~12.288~ & ~0.341~ & ~0.028~ & ~0.007~ & ~0.004~ & ~0.008~ & ~0.008~ & ~0.007~ & ~0.012~ & ~0.009~ & ~0.005~ & ~0.001~ & ~0.012~ \\
      ~5~ & ~-~ & ~-~ & ~26.047~ & ~0.343~ & ~0.020~ & ~0.010~ & ~0.026~ & ~0.024~ & ~0.021~ & ~0.031~ & ~0.021~ & ~0.013~ & ~0.396~ & ~0.055~ \\
      ~6~ & ~-~ & ~-~ & ~409.079~ & ~16.742~ & ~0.122~ & ~0.031~ & ~0.268~ & ~0.232~ & ~0.103~ & ~0.085~ & ~0.066~ & ~0.046~ & ~4.062~ & ~0.808~ \\
      ~7~ & ~-~ & ~-~ & ~-~ & ~593.666~ & ~0.670~ & ~0.104~ & ~1.896~ & ~1.967~ & ~1.179~ & ~0.533~ & ~0.400~ & ~0.255~ & ~3.660~ & ~94.429~ \\
      ~8~ & ~-~ & ~-~ & ~-~ & ~-~ & ~2.579~ & ~0.149~ & ~12.851~ & ~10.589~ & ~5.566~ & ~3.088~ & ~2.655~ & ~1.531~ & ~-~ & ~-~ \\
      ~9~ & ~-~ & ~-~ & ~-~ & ~-~ & ~13.958~ & ~0.339~ & ~468.581~ & ~422.396~ & ~102.687~ & ~61.973~ & ~60.465~ & ~19.984~ & ~-~ & ~-~ \\
      ~10~ & ~-~ & ~-~ & ~-~ & ~-~ & ~64.208~ & ~1.318~ & ~-~ & ~-~ & ~-~ & ~-~ & ~-~ & ~1189.290~ & ~-~ & ~-~ \\
      \hline
    \end{tabular}
}
  \end{center}
\end{sidewaystable}



\chapter{Conclusão}
\label{cap:concl}
TODO: Conclusão do trabalho
% % conclusao
Neste trabalho foram apresentados modelos de Programação por
Restrições para ordenação por reversões e ordenação por reversões e
transposições, baseados na teoria do Problema de Satisfação de
Restrições e na teoria do Problema de Otimização com Restrições,
seguindo a linha de pesquisa utilizada por Dias e
Dias~\cite{DiasDias*2009}.

Fizemos a comparação dos modelos de programação por restrições para os
problemas de ordenação por transposições, descrito por Dias e
Dias~\cite{DiasDias*2009}, ordenação por reversões e ordenação por
reversões e transposições com os modelos de programação linear inteira
para os mesmos problemas descritos em Dias e
Souza~\cite{DiasSouza*2007}.





\appendix

%% Revisão Bibliográfica
% \chapter{Revisão Bibliográfica} 

% Neste apêndice apresentamos resumos de trabalhos relevantes em
% alinhamento múltiplo de seqüências.

% \input{apendice/revisao/DuretEtAl2000}
% 
% \input{apendice/revisao/Notredame2002}
% 
% \input{apendice/revisao/Notredame2007}
% 
% \input{apendice/revisao/WallaceEtAl2005}
% 
% \input{apendice/revisao/BAliBASE}
% 
% \input{apendice/revisao/ThompsonEtAl1999}
% 
% \input{apendice/revisao/Progressivo}
% 
% \input{apendice/revisao/ClustalW}
% 
% \input{apendice/revisao/DiAlign}
% 
% \input{apendice/revisao/ProbCons}
% 
% \input{apendice/revisao/SAGA}
% 
% \input{apendice/revisao/PRRP}
% 
% \input{apendice/revisao/DbClustal}
% 
% \input{apendice/revisao/MCoffee}
% 
% \input{apendice/revisao/MeidanisSetubal1995}
% 
% %% Estado da Arte
% \include{apendice/survey/Survey}
% 
% %% Vocabulário
% \include{apendice/vocabulario/vocabulario}

\bibliographystyle{plain}
\bibliography{tese}

\end{document}
